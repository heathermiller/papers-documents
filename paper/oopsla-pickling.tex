%-----------------------------------------------------------------------------
%
%               Template for sigplanconf LaTeX Class
%
% Name:         sigplanconf-template.tex
%
% Purpose:      A template for sigplanconf.cls, which is a LaTeX 2e class
%               file for SIGPLAN conference proceedings.
%
% Guide:        Refer to "Author's Guide to the ACM SIGPLAN Class,"
%               sigplanconf-guide.pdf
%
% Author:       Paul C. Anagnostopoulos
%               Windfall Software
%               978 371-2316
%               paul@windfall.com
%
% Created:      15 February 2005
%
%-----------------------------------------------------------------------------


\documentclass[preprint,10pt]{sigplanconf}

% The following \documentclass options may be useful:
%
% 10pt          To set in 10-point type instead of 9-point.
% 11pt          To set in 11-point type instead of 9-point.
% authoryear    To obtain author/year citation style instead of numeric.

\usepackage{amsmath}

\begin{document}

\conferenceinfo{WXYZ '05}{date, City.}
\copyrightyear{2005}
\copyrightdata{[to be supplied]}

\titlebanner{banner above paper title}        % These are ignored unless
\preprintfooter{short description of paper}   % 'preprint' option specified.

\title{Object-Oriented Pickler Combinators}
\subtitle{Subtitle Text, if any}

\authorinfo{Heather Miller}
           {EPFL}
           {heather.miller@epfl.ch}
\authorinfo{Philipp Haller}
           {Typesafe, Switzerland}
           {philipp.haller@typesafe.com}
\authorinfo{Eugene Burmako}
           {EPFL}
           {eugene.burmako@epfl.ch}
\authorinfo{Martin Odersky}
           {EPFL}
           {martin.odersky@epfl.ch}

\maketitle

\begin{abstract}
This is the text of the abstract.
\end{abstract}

\category{CR-number}{subcategory}{third-level}

\terms
term1, term2

\keywords
keyword1, keyword2

\section{Introduction}

[Page limit: 14 pages including references.]
Serialization or pickling, i.e., persisting runtime objects by
converting them into a binary or text representation is ubiquitous in
distributed programming.

There has never been a Scala-specific solution to serialization which
can support certain aspects of Scala's type system, nor which can
generate serialization-related boilerplate at compile-time. Even the
fastest Java serialization frameworks must generate all
pickler-related code at runtime, which in preliminary benchmarks
amounts to a factor 10 slow-down over a naive but fully-static pickler
combinator-based approach.

The goal of this project is a new framework for pickling (or
serialization). The idea is to automatically generate pickler
combinators at compile-time.

The main contribution would be to extend existing approaches to
pickler combinators with support for object-oriented mechanisms, such
as subclassing.

\paragraph{Guiding principles:}

Scala Pickling should be:
\begin{itemize}
  \item more typesafe than Java serialization.
  \item faster than Java serialization.
  \item more extensible than Java serialization.
  \item but should not be more complicated to think about than Java serialization.
\end{itemize}

\subsection{Related Work}

Figure~\ref{fig:comparison} compares the main pickling/serialization
frameworks with respect to type-safety, object-orientation, type
extensibility, and format extensibility.

\begin{figure}[t]
\begin{tabular}{|l|l|l|l|l|l|l|}
\hline
Framework    & Type-safety     & Object-oriented & Type extensibility
& Format Extensibility \\
\hline
Java Serialization      & Java-only & yes & no & no \\
Kryo                         & Java-only & yes & yes & no \\
Pickler combinators & yes           & no & yes & (yes) \\
This paper                & yes          & yes & yes & yes \\
\hline
\end{tabular}

\caption{Comparing serialization frameworks}\label{fig:comparison}
\end{figure}


\subsection{Contributions}

\begin{itemize}

\item A new pickling framework for object-oriented languages that (a)
  is fast through compile-time generated picklers, (b) enables
  retrofitting pickling support to existing types retroactively, (c)
  supports pluggable pickling formats, and (d) does not require
  changes to the underlying VM.

  The framework is thus extensible in several dimensions: first,
  pickling can be enabled for classes that have not been prepared
  beforehand (through extending a specific interface, for
  example). Second, new pickling formats (JSON, XML, Protobuf, etc.)
  can be added and selected modularly.

\item Our approch extends pickler combinators (a well-established
  approach in the functional programming
  community~\cite{Kennedy,Elsman}) to support core concepts of
  object-oriented programming, namely open class hierarchies and
  ad-hoc polymorphism (runtime dispatch).

\item To the best of our knowledge we are the first to extend pickler
  combinators with pluggable pickling formats.

\item We present a complete implementation of our approach in
  Scala.\footnote{See
    \texttt{http://github.com/heathermiller/scala-pickling/}} We have
  evaluated our framework by comparing its performance with the native
  serialization support of the Java Virtual Machine, as well as the
  Kryo serialization framework~\cite{Kryo} for Java. In the context of
  a suite of microbenchmarks, our framework outperforms Java
  serialization by a factor of X, and Kryo by a factor of Y. We have
  also integrated our pickling framework in Spark~\cite{Zaharia2010}
  and Akka~\cite{Akka}, and found that on representative applications
  our pickling framework improves performance by $X \%$ (Spark) and $Y
  \%$ (Akka) on average without requiring changes in user code.

\end{itemize}


\section{Background}

The design and implementation of our pickling framework leverages
several advanced features of the Scala programming language. This
section introduces everything that is required to understand the rest
of the paper. Apart from this section the reader is expected to be
familiar with a typical statically-typed, class-based object-oriented
programming language such as Java or C\#.

\subsection{Implicits}

\paragraph{Importing implicit values}

\subsection{Macros}

\section{Overview}

\subsection{Usage}

Ideally, a user would use the Scala Pickling framework as follows,

\begin{verbatim}
class Person(name: String, age: Int)

val p: Person = new Person("Bob", 61)
val bytes = p.pickle
\end{verbatim}\noindent

That is, the assumption is that you would be able to directly call a
pickle method on an object where pickle might not be defined. In this
case, most of the time, the framework should be able to generate all
relevant pickler combinators for your arbitrary object at
compile-time.

Furthermore, the user should also be able to select pickle formats. By
default, a Scala Binary format would be used, but it's planned to add
support for JSON and Protobuf as well. A user would select an
alternate pickle format as follows:

\begin{verbatim}
import scala.pickling.JSONFormat

...
val bytes = p.pickle
\end{verbatim}\noindent

That is, alternate pickling formats would be represented as implicit
values which simply need to be in scope. Thus, extensibility for
alternate back-ends should be quite straightforward.


\section{Generating Fast Picklers}

\subsection{Front-end}


\section{Pluggable Pickling Formats}

\section{Implementation}

\section{Experimental Evaluation}

\appendix
\section{Appendix Title}

This is the text of the appendix, if you need one.

\acks

Acknowledgments, if needed.

% We recommend abbrvnat bibliography style.

\bibliographystyle{abbrvnat}

% The bibliography should be embedded for final submission.

%\begin{thebibliography}{}
%\softraggedright

%\bibitem[Smith et~al.(2009)Smith, Jones]{smith02}
%P. Q. Smith, and X. Y. Jones. ...reference text...

%\end{thebibliography}

\bibliography{bib}

\end{document}
