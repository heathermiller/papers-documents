This section formulates progress properties. The main
Theorem~\ref{thm:finite-mat} states that materialization requests are
satisfied after a finite number of reduction steps in so-called
``responsive configurations.''

In the following we assume a {\em fair scheduling} property which
ensures that in a well-formed configuration $H~|~M$, each message $h
\leftarrow m \in M$ is eventually received by host $h$. Fair
scheduling is also assumed in other models of distributed computing
like actors~\cite{Actors,Talcott}. Formally, fair scheduling is
defined as follows:

\begin{defn}[Fair Scheduling]\label{def:fair-scheduling}
  Let $\Sigma \vdash H_0~|~M_0$ and $h \leftarrow m \in M_0$ where
  $\Sigma \vdash m$.

  Then $\exists n > 0$ such that $H_0~|~M_0 \twoheadrightarrow^*
  \ldots \twoheadrightarrow^* H_n~|~M_n \twoheadrightarrow H'~|~M'$
  where \\$\forall i \in \{ 0, \ldots, n-1 \}.~H_i~|~M_i
  \twoheadrightarrow^* H_{i+1}~|~M_{i+1}$ after a finite number of
  reduction steps, \\$\forall i \in \{ 1, \ldots, n
  \}.~\mathit{Blocked}(h, H_i)$, and $H_n~|~M_n \twoheadrightarrow
  H'~|~M'$ by rule \textsc{R-Process} or \textsc{R-Process-Val} such
  that $M_n = \hat{M} \uplus \{ h \leftarrow m \}$, $(t, \sigma)^h \in
  H_n$, $process(h, m, \sigma) = (t', M'', \sigma')$, and $M' =
  \hat{M} \uplus M''$.
\end{defn}

The predicate $\mathit{Blocked}(h, H)$ expresses the fact that host
$h$ cannot be reduced further using local reduction; i.e., host $h$ is
``blocked'' and is therefore able to process incoming
messages. Formally, $\mathit{Blocked}$ is defined as follows:

\begin{defn}[Blocked Host]\label{def:blocked-host}
  A host $h$ is {\em blocked} in a set of hosts $H$, written
  $\mathit{Blocked}(h, H)$, {\em iff} $(t, \sigma)^h \in H$ such that
  $t = E[\texttt{await}(\iota)]$ or $t = E[v]$.
\end{defn}

Although our focus is the establishment of desirable progress
properties for distributed reduction, it is necessary to consider the
following {\em strong normalization} property of single-host
reductions. For this, we consider the reduction relation $\leadsto$
defined as the subset of the reduction relation $\twoheadrightarrow$
excluding reduction rules \textsc{R-Process} and
\textsc{R-Process-Val}.

\begin{lem}[Single-Host Strong Normalization]\label{lem:strong-normal}
  Let $\Sigma \vdash H~|~M$ where $H = \{ (t, \sigma)^h \} \cup
  H'$. Then $H~|~M \leadsto^* \{ (t', \sigma')^h \} \cup H'~|~M'$
  after a finite number of reduction steps and either $t'$ is a value
  or $t' = E[\texttt{await}(\iota)]$.
\end{lem}

Note that in the above reduction, only a single host $h$ is
reduced. Furthermore, the set of in-flight messages may change, e.g.,
by applying rule \textsc{R-Send}. A proof of
Lemma~\ref{lem:strong-normal} is outside the scope of this
paper. However, our core language is, fundamentally, not more
expressive than the simply typed lambda calculus, for which strong
normalization holds.
