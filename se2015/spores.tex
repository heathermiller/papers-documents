\documentclass[english]{lni}
\let\ifpdf\relax

\IfFileExists{latin1.sty}{\usepackage{latin1}}{\usepackage{isolatin1}}

\usepackage{graphicx}

\author{
	Heather Miller$^*$, Philipp Haller$^\dagger$ \\
	\\
	EPFL, Switzerland$^*$ \\
	KTH Royal Institute of Technology, Sweden$^\dagger$
}
\title{A Type-Based Foundation for Closure-Passing in the Age of Concurrency and Distribution}
\begin{document}
\maketitle

\begin{abstract}
Functional programming (FP) is regularly touted as the way forward for
bringing parallel, concurrent, and distributed programming to the mainstream.
However, despite this established viewpoint, reliably distributing function
closures over a network, or using them in concurrent environments nonetheless
remains a challenge across FP and OO languages. Our work on Spores takes a
step towards more principled distributed and concurrent programming by
introducing a new closure-like abstraction and type system that can
guarantee closures to be serializable, thread-safe, or have custom
user-defined properties. In ongoing work we explore the combination of
Spores and Scala Pickling to provide a common substrate for type-safe,
performant data-intensive applications.
\end{abstract}

\vspace{-10mm}
\section{Motivation}
\vspace{-4mm}
With the growing trend towards decentralized computing--cloud computing,
mobile applications, and big data--distributed programming has entered the
mainstream. Meanwhile, functional programming (FP) has been
gaining traction, as is evidenced by the ongoing
trend of imperative OO languages being extended
with FP features, such as lambdas in \mbox{Java 8}, \mbox{C++ 11}, and
\mbox{VB 9}, the perceived importance of FP in empirical
studies on software developers, and the popularity of FP
massive online open courses (MOOCs).\footnote{In the interest
of space, references are omitted from this extended abstract, and are found
in the paper: Heather Miller, Philipp Haller, and Martin Odersky. Spores: A
Type-Based Foundation for Closures in the Age of Concurrency and Distribution.
In ECOOP, volume 8586, LNCS, pages 308--333. Springer, 2014.}

One reason for the rise in popularity of FP features within object-oriented
communities is the observation that
a functional style simplifies reasoning about data in
parallel, concurrent, and distributed code. Distributed data-parallel
frameworks like MapReduce and Spark are designed
around FP patterns where closures are transmitted across cluster nodes
to large-scale persistent datasets. As a result of the ``big data''
revolution, these frameworks have become very popular, in turn further
highlighting the need to be able to reliably and safely serialize and transmit
closures over the network.

However, for both OO and FP languages, there still exist
numerous hurdles for even these most basic functional
building blocks to overcome in order to be reliable and easy to
reason about in a concurrent or distributed setting. Closure-related
hazards related to concurrency and distribution include:
(1) accidental capture of non-serializable variables;
(2) transitive references that inadvertently hold
on to large object graphs, creating memory leaks; (3) capturing
references to mutable objects, leading to race conditions in a concurrent
setting; (4) unknowingly accessing object members that are not constant,
leading to semantic inconsistencies when closures are distributed.

% which in a distributed setting can have logically different
% meanings on different machines.

\vspace{-7mm}
\section{Spores}
\vspace{-5mm}
As a step towards more principled {\em function-passing style} we
introduce a type-based foundation for closures, {\em spores}.
Spores are a closure-like abstraction and type system which is designed to
avoid typical hazards of closures. By including type information of captured
variables in the type of a spore, we enable the expression of type-based
constraints for captured variables. We show that this approach can be made
practical by automatically synthesizing refinement types using macros, and
by leveraging local type inference. Using type-based constraints, spores
allow expressing a variety of ``safe'' closures.

To express safe closures with transitive properties such as guaranteed
serializability, or closures capturing only deeply immutable types, spores
support type constraints based on type classes which enforce transitive
properties. In addition, implicit macros in Scala enable integration with type
systems that enforce transitive properties using generics or annotated types.
Spores also support user-defined type constraints. Finally, by principle of a
type-based approach, spores can potentially benefit from optimization, further
safety checks via type system extensions, and verification opportunities.

% A spore is a closure with a specific shape that dictates how the environment
% of a spore is declared, as well as a refined type indicating what the spore
% captures. Syntactically, a spore consists of two parts:

% \begin{itemize}
% \item {\bf the spore header}, composed of a list of value definitions.
% \item {\bf the spore body} (sometimes referred to as the ``spore closure''), a regular closure.
% \end{itemize}

% \begin{figure}[h!]
% \centering
% \includegraphics{syntax}
% \caption{The syntactic shape of a spore.}
% \label{fig:spore-shape}
% \end{figure}

% The characteristic property of a spore is that the {\em spore body} is only
% allowed to access its parameter, the values in the spore header, as well as
% top-level singleton objects (public, global state). In particular, the spore
% closure is not allowed to capture variables in the environment. Only an
% expression on the right-hand side of a value definition in the spore header is
% allowed to capture variables.

% By enforcing this shape, the environment of a spore is always declared
% explicitly in the spore header, which avoids accidentally capturing
% problematic references.

% \begin{figure}[h!]
% \centering
% \includegraphics{type}
% \caption{The syntactic shape of a spore.}
% \label{fig:spore-shape}
% \end{figure}

\vspace{-7mm}
\section{Formalization and Type System}
\vspace{-5mm}
In our accompanying publication and technical report,
we present a formalization of spores with type constraints in the context of a
standard typed lambda calculus with records, and prove soundness of the type
system.

\vspace{-7mm}
\section{Empirical Studies}
\vspace{-5mm}
We evaluate the practicality and the benefits of using spores as an
alternative to normal closures in Scala. First, we measure the impact of
introducing spores in existing programs. Second, we evaluate the utility
and the syntactic overhead of spores in a large code base of applications
based on the Apache Spark framework for big data analytics.

\vspace{-5mm}
\paragraph{Using spores instead of closures}
We analyzed a number of real Scala programs: (1) general, closure-heavy
code, taken from the popular MOOC on
FP Principles in Scala, (2) parallel programs using parallel
collections, and (3) distributed programs based on Spark. Our
results show that of all closures, 90\% could be converted without extra
effort. For those that had to be manually converted, on average only 1.4 LOC
had to be changed. This suggests that programs using closures in non-trivial
ways can typically be converted to using spores with little effort.

\vspace{-5mm}
\paragraph{Spores and Apache Spark}
To evaluate both benefit and overhead of using spores in real distributed
programs, we studied the codebases of 7 large open-source
applications using Spark. Our results show that of all closures
passed to Spark's \verb|RDD.map| method, about 67.2\% do not capture any
variables; these closures could be automatically converted to spores. The
remaining 32.8\% of closures capture
1.39 variables on average. This indicates that unchecked patterns for
serializable closures are widespread in real programs, and that benefiting
from the static guarantees of spores would require only little syntactic overhead.

% \bibliography{bib}



\end{document}
