
%%%%%%%%%%%%%%%%%%%%%%% file typeinst.tex %%%%%%%%%%%%%%%%%%%%%%%%%
%
% This is the LaTeX source for the instructions to authors using
% the LaTeX document class 'llncs.cls' for contributions to
% the Lecture Notes in Computer Sciences series.
% http://www.springer.com/lncs       Springer Heidelberg 2006/05/04
%
% It may be used as a template for your own input - copy it
% to a new file with a new name and use it as the basis
% for your article.
%
% NB: the document class 'llncs' has its own and detailed documentation, see
% ftp://ftp.springer.de/data/pubftp/pub/tex/latex/llncs/latex2e/llncsdoc.pdf
%
%%%%%%%%%%%%%%%%%%%%%%%%%%%%%%%%%%%%%%%%%%%%%%%%%%%%%%%%%%%%%%%%%%%


\documentclass[runningheads,a4paper]{llncs}

\usepackage{amssymb}
\setcounter{tocdepth}{3}
\usepackage{graphicx}

\usepackage{url}
\urldef{\mails}\path|firstname.lastname@epfl.ch|
\urldef{\mailsa}\path|firstname.lastname@typesafe.com|
% \urldef{\mailsa}\path|heather.miller@epfl.ch|
% \urldef{\mailsb}\path|philipp.haller@typesafe.com|
\newcommand{\keywords}[1]{\par\addvspace\baselineskip
\noindent\keywordname\enspace\ignorespaces#1}

\begin{document}

\mainmatter  % start of an individual contribution

% first the title is needed
\title{Spores: Elevating Scala to the Cloud}
% a short form should be given in case it is too long for the running head
\titlerunning{Spores: Elevating Scala to the Cloud}

% the name(s) of the author(s) follow(s) next
%
% NB: Chinese authors should write their first names(s) in front of
% their surnames. This ensures that the names appear correctly in
% the running heads and the author index.
%
% \author{Heather Miller%
% \and Philipp Haller}
\author{Heather Miller\inst{1} \and Philipp Haller\inst{2} \and Martin Odersky\inst{1}}
%
% \authorrunning{Lecture Notes in Computer Science: Authors' Instructions}
% (feature abused for this document to repeat the title also on left hand pages)

% the affiliations are given next; don't give your e-mail address
% unless you accept that it will be published
% \institute{EPFL, Switzerland\\
% \mailsa\\
% \mailsb\\}
\institute{EPFL, Switzerland\\
\mails
\and
Typesafe, Inc.\\
\mailsa}

%
% NB: a more complex sample for affiliations and the mapping to the
% corresponding authors can be found in the file "llncs.dem"
% (search for the string "\mainmatter" where a contribution starts).
% "llncs.dem" accompanies the document class "llncs.cls".
%

\toctitle{Lecture Notes in Computer Science}
\tocauthor{Authors' Instructions}
\maketitle


\begin{abstract}
The abstract should summarize the contents of the paper and should
contain at least 70 and at most 150 words. It should be written using the
\emph{abstract} environment.
\keywords{We would like to encourage you to list your keywords within
the abstract section}
\end{abstract}


\section{Introduction}
Citing \cite{CloudHaskell}

\bibliographystyle{abbrv}
\bibliography{bib}

% \begin{thebibliography}{4}

% \bibitem{jour} Smith, T.F., Waterman, M.S.: Identification of Common Molecular
% Subsequences. J. Mol. Biol. 147, 195--197 (1981)

% \bibitem{lncschap} May, P., Ehrlich, H.C., Steinke, T.: ZIB Structure Prediction Pipeline:
% Composing a Complex Biological Workflow through Web Services. In: Nagel,
% W.E., Walter, W.V., Lehner, W. (eds.) Euro-Par 2006. LNCS, vol. 4128,
% pp. 1148--1158. Springer, Heidelberg (2006)

% \bibitem{book} Foster, I., Kesselman, C.: The Grid: Blueprint for a New Computing
% Infrastructure. Morgan Kaufmann, San Francisco (1999)

% \bibitem{proceeding1} Czajkowski, K., Fitzgerald, S., Foster, I., Kesselman, C.: Grid
% Information Services for Distributed Resource Sharing. In: 10th IEEE
% International Symposium on High Performance Distributed Computing, pp.
% 181--184. IEEE Press, New York (2001)

% \bibitem{proceeding2} Foster, I., Kesselman, C., Nick, J., Tuecke, S.: The Physiology of the
% Grid: an Open Grid Services Architecture for Distributed Systems
% Integration. Technical report, Global Grid Forum (2002)

% \bibitem{url} National Center for Biotechnology Information, \url{http://www.ncbi.nlm.nih.gov}

% \end{thebibliography}




\end{document}
