% This is LLNCS.DEM the demonstration file of
% the LaTeX macro package from Springer-Verlag
% for Lecture Notes in Computer Science,
% version 2.4 for LaTeX2e as of 16. April 2010
%
\documentclass{llncs}
%
\usepackage{makeidx}  % allows for indexgeneration
\usepackage{url}
\usepackage{todonotes}
\usepackage{listings}
\usepackage{bcprules}
\usepackage{fontspec}
\usepackage{fancyvrb}

\usepackage{microtype}
\usepackage{times}
\usepackage[utf8]{inputenc}
\usepackage[english]{babel}
\usepackage{graphicx}
\usepackage{xcolor}
\usepackage{lipsum}
%%%%********************************************************************
% fancy quotes
\definecolor{quotemark}{gray}{0.7}
\makeatletter
\def\fquote{%
    \@ifnextchar[{\fquote@i}{\fquote@i[]}%]
           }%
\def\fquote@i[#1]{%
    \def\tempa{#1}%
    \@ifnextchar[{\fquote@ii}{\fquote@ii[]}%]
                 }%
\def\fquote@ii[#1]{%
    \def\tempb{#1}%
    \@ifnextchar[{\fquote@iii}{\fquote@iii[]}%]
                      }%
\def\fquote@iii[#1]{%
    \def\tempc{#1}%
    \vspace{1em}%
    \noindent%
    \begin{list}{}{%
         \setlength{\leftmargin}{0.1\textwidth}%
         \setlength{\rightmargin}{0.1\textwidth}%
                  }%
         \item[]%
         \begin{picture}(0,0)%
         \put(-15,-5){\makebox(0,0){\scalebox{3}{\textcolor{quotemark}{``}}}}%
         \end{picture}%
         \begingroup\itshape}%
 %%%%********************************************************************
 \def\endfquote{%
 \endgroup\par%
 \makebox[0pt][l]{%
 \hspace{0.8\textwidth}%
 \begin{picture}(0,0)(0,0)%
 \put(15,15){\makebox(0,0){%
 \scalebox{3}{\color{quotemark}''}}}%
 \end{picture}}%
 \ifx\tempa\empty%
 \else%
    \ifx\tempc\empty%
       \hfill\rule{100pt}{0.5pt}\\\mbox{}\hfill\tempa,\ \emph{\tempb}%
   \else%
       \hfill\rule{100pt}{0.5pt}\\\mbox{}\hfill\tempa,\ \emph{\tempb},\ \tempc%
   \fi\fi\par%
   \vspace{0.5em}%
 \end{list}%
 }%
 \makeatother
 %%%%********************************************************************

% Member sequences
\newcommand{\seq}[1]{\overline{#1}}

% arrays
\newcommand{\ba}{\begin{array}}
\newcommand{\ea}{\end{array}}
\newcommand{\bda}{\[\ba}
\newcommand{\eda}{\ea\]}
\newcommand{\ei}{\end{array}}
\newcommand{\bcases}{\left\{\begin{array}{ll}}
\newcommand{\ecases}{\end{array}\right.}
\newcommand{\sporeurl}{URL withheld for the sake of anonymity of review.}
% spacing
\newcommand{\gap}{\quad\quad}
\newcommand{\biggap}{\quad\quad\quad}
\newcommand{\nextline}{\\ \\}
\newcommand{\htabwidth}{0.5cm}
\newcommand{\tabwidth}{1cm}
\newcommand{\htab}{\hspace{\htabwidth}}
\newcommand{\tab}{\hspace{\tabwidth}}
\newcommand{\linesep}{\ \hrulefill \ \smallskip}

\lstdefinelanguage{Scala}%
{morekeywords={abstract,case,catch,char,class,%
    def,else,extends,final,%
    if,import,%
    match,module,new,null,object,override,package,private,protected,%
    public,return,super,this,throw,trait,try,type,val,var,with,implicit,%
    macro,sealed,%
  },%
  sensitive,%
  morecomment=[l]//,%
  morecomment=[s]{/*}{*/},%
  morestring=[b]",%
  morestring=[b]',%
  showstringspaces=false%
}[keywords,comments,strings]%

\lstset{language=Scala,%
  mathescape=true,%
  columns=[c]fixed,%
  basewidth={0.5em, 0.40em},%
  basicstyle=\tt,%
  xleftmargin=0.0cm
}

\setmainfont[Ligatures=TeX]{Times}

%
\begin{document}

\VerbatimFootnotes
\setmonofont[Scale=0.8,BoldFont={Consolas Bold}]{Consolas}

%
\mainmatter              % start of the contributions

% Title ideas:
% - Spores: Reliably Distribute Your Closures
% - Towards Function-Passing Style via Reliable
% - Spores: Function-Passing Style via Distributable Closures
% - Spores: Closures for Distributed and Concurrent Programming
% - Spores: Enabling Function-Passing Style via Distributable Closures
% - Spores: Safe Closures for Distributed and Concurrent Programming
% - Spores: Closures with Context Bounds for Safe Distributed and Concurrent Programming
% - Spores: Controlling the Environment of Closures for Safe Distributed and Concurrent Programming
% - Spores: Closure Environment Constraints for Safe Distributed and Concurrent Programming
% - Spores: Closures for Reliable Distributed and Concurrent Programming

% Things to perhaps mention:
% - spores
% - function-passing style
% - safe
% - type-directed
% - distributed
% - concurrent

\title{Towards Lambda the Ultimate Distributive}
\subtitle{Safe Closures for Reliable Distributed and Concurrent Programming}
\titlerunning{Towards Lambda the Ultimate Distributive: Safe Closures for Reliable Distributed and Concurrent Programming}  % abbreviated title (for running head)
%                                     also used for the TOC unless
%                                     \toctitle is used
%
% \author{Heather Miller\inst{1} \and Philipp Haller\inst{2}
% \and Martin Odersky\inst{1}}
%
% \authorrunning{Heather Miller et al.} % abbreviated author list (for running head)
%
%%%% list of authors for the TOC (use if author list has to be modified)
% \tocauthor{Heather Miller, Philipp Haller, and Martin Odersky}
%
% \institute{EPFL, Switzerland\\
% \and
% Typesafe, Switzerland}

\maketitle              % typeset the title of the contribution

\begin{abstract}

Functional programming (FP) is regularly touted as the way forward for bringing
parallel, concurrent, and distributed programming to the mainstream. The
popularity of the rationale behind this viewpoint (immutable data transformed
by function application) has even lead to a number of object-oriented (OO)
programming languages adopting functional features such as lambdas
(functions) and thereby function closures.
However, despite this established viewpoint of FP as an enabler, reliably
distributing function closures over a network, or using them in concurrent
environments nonetheless remains a challenge across FP and OO languages.
This paper takes a step towards more principled distributed and concurrent
programming by introducing a new closure-like abstraction and type system,
called {\em spores}, that can guarantee closures to be serializable, thread-safe,
or even have general, custom user-defined properties.
We prove our type system sound, implement our approach for Scala\footnote{\sporeurl}, evaluate
its practicality through an small empirical study, and show the power of these
guarantees through a case analysis of new distributed and concurrent
frameworks that this safe foundation for migratable closures enables.

% == 150-word abstract for the submission page ==
% Functional programming is regularly touted as the way forward for bringing
% parallel, concurrent, and distributed programming to the mainstream. The
% popularity of the rationale behind this viewpoint (immutable data transformed
% by function application) has even lead to a number of object-oriented
% programming languages adopting functional features such as lambdas
% and thereby function closures.
% However, despite this established viewpoint of FP as an enabler, reliably
% distributing closures over a network, or using them in concurrent
% environments nonetheless remains a challenge across FP and OO languages.
% This paper takes a step towards more principled distributed and concurrent
% programming by introducing a new closure-like abstraction and type system,
% called spores, that can guarantee closures to be serializable, thread-safe,
% or even have custom user-defined properties.
% We prove our type system sound, implement our approach for Scala, and show the
% power of these guarantees through a case analysis of new distributed/concurrent
% frameworks that this safe foundation for migratable closures enables.

\keywords{functions, closures, distributed programming, concurrent programming, type systems}
\end{abstract}
%
\section{Introduction}

With the growing trend towards cloud computing, mobile applications, big data,
distributed programming has entered the mainstream. The
traditional view of software development as being focused on a program running
on a single machine, interacting directly with the user, has become largely
obsolete. Popular paradigms in software engineering such as software as a
service (SaaS), RESTful services, or the rise of multitudes of different
models and systems for big data processing and interactive analytics, evidence
this trend. Whether we consider a cluster of hundreds of commodity machines
churning through a massive data-parallel job, or a smartphone interacting with
a social network, all are ``distributed'' jobs, and all share the need to
interact in typically asynchronous, reactive ways with other clients or
services.

Meanwhile, at the same time, functional programming has been undeniably
gaining traction in recent years, as is evidenced by the ongoing trend of
traditionally object-oriented or imperative languages being extended with
functional features, such as lambdas in \mbox{Java 8}~\cite{JavaLambdas},
C++11~\cite{CplusplusLambas}, and Visual Basic 9~\cite{Meijer}, the perceived
importance of functional programming in general empirical studies on software
developers~\cite{PLAdoption}, and the measurable popularity of functional
programming massively online open courses (MOOCs)~\cite{ICSEMOOC}.

One reason for the rise in popularity of functional programming languages and
features within object-oriented communities is the basic philosophy of
transforming immutable data by applying first-class functions, and the
observation that this functional style simplifies reasoning about data in
parallel, concurrent, and distributed code. A popular and well-understood
example of this style of programming for which many popular frameworks have
come to fruition is functional data-parallel programming, where the
fundamental idea is to distribute data across nodes/threads and to perform the
same task on the different pieces of this distributed data in parallel.
Examples across functional and object-oriented paradigms include \mbox{Java
8}'s monadic-style optionally parallel collections~\cite{JavaLambdas}, Scala's
parallel~\cite{ScalaParColls} and concurrent dataflow~\cite{FlowPools}
collections, Data Parallel Haskell~\cite{DataParallelHaskell}, Concurrent
Collections (CnC)~\cite{CnC} Nova~\cite{Nova}, and Haskell's \verb|Par|
monad~\cite{HaskellPar} to name a few.

% Data-parallel programming has been established as a particularly successful
% and well-understood form of parallel programming across both programming
% paradigms (both functional and object-oriented proposals exist) and domains
% (high-performance computing, big data analytics, GPU programming, ...). A
% ``function-passing'' style of data-parallel programming, based on immutable
% data and first-class functions, enables more opportunities for optimization,
% safety, and verification opportunities~\cite{Nova}.

In the context of distributed programming, data-parallel frameworks like
Hadoop or MapReduce~\cite{MapReduce} and Spark~\cite{Spark} are designed
around functional patterns where closures encapsulating computations are
transmitted over the network to large-scale persistent datasets stored on
different nodes in the cluster. As a result of the ``big data'' revolution,
these frameworks have become very popular, in turn further highlighting the
need to be able to reliably and safely serialize and transmit closures over
the network. % making the robust distribution of computations and % closures
essential.

Frameworks like Spark embody a pattern that we like to call ``Lambda The
Ultimate Distributive'' where closures are shipped to the data such that (a)
data to-be-processed remotely is passed as arguments, (b) data that needs to
be shipped with the closure is captured and serialized, and (c) services local
to a node are either passed as additional arguments by the framework or looked
up.

% have to be pure or can't have access to mutable things

{\bf However, there's trouble in paradise.}~For both object-oriented and
functional languages, there still exist numerous hurdles at the language-level
for even these most basic functional building blocks, closures, to overcome in
order to be reliable and easy to reason about in a concurrent or distributed
setting.

In order to distribute closures, one must be able to serialize them -- a goal
that remains tricky to reliably achieve not only hard in object-oriented
languages but also in pure functional languages like Haskell:

\begin{lstlisting}
  sendFunc :: SendPort (Int -> Int) -> Int -> ProcessM ()
  sendFunc p x = sendChan p (\y -> x + y + 1)
\end{lstlisting}

In this Haskell example, the function sent on the channel,
\verb|(\y -> x + y + 1)|,
is a closure that captures its free variables, in this case \verb|x|.
As one might expect, to serialize a function value means that one must also
serialize its free variables. However, in this case, the types of these free
variables are unrelated to the type of the function value, so it is entirely
unclear how to serialize them.

In languages like Java or C\# on the other hand, serialization is solved
differently -- the runtime environment is designed to be able to serialize any
object, reflectively. While this ``universal'' serialization might seem to
solve the problem of languages like Haskell that cannot rely on such a
mechanism, serializing closures nonetheless remains surprisingly error-prone.
For example, attempting to serialize a closure with transitive references to
objects that are not marked as serializable will crash at runtime, typically
with no compile-time checks whatsoever. The kicker is that it's remarkably
easy to accidentally create such a problematic transitive reference,
especially in an object-oriented language.

For example, consider the following use of a distributed collection in Scala
with higher-order functions \verb|map| and \verb|reduce| (using Spark):

\begin{lstlisting}
class MyCoolRddApp {
  val log = new Log(...)
  def shift(p: Int): Int = ...
  ...
  def work(rdd: RDD[Int]) {
    rdd.map(x => x + shift(x)).reduce(...)
  }
}
\end{lstlisting}

In this example, the closure \verb|(x => x + shift(x))| is passed to the
\verb|map| method of the distributed collection \verb|rdd| which requires
serializing the closure (as, in Spark, parts of the data structure reside on
different machines). However, calling \verb|shift| inside the closure invokes
a method on the enclosing object \verb|this|. Thus, the closure is capturing,
and must therefore serialize, \verb|this|. If \verb|Log|, referred to from
\verb|this|, is not serializable, this will fail at runtime.

In fact, closures suffer not only from the problems shown in these two
examples; there are numerous more hazards that manifest {\em across
programming paradigms}. To provide a complete glimpse, closure-related hazards
related to concurrency and distribution include:

% Unfortunately, besides the above mentioned problems, closures exhibit a variety of additional hazards, across programming paradigms.

\begin{itemize}
\item free variables that are not serializable;
\item accidental references to \verb|this| or other enclosing objects that are not serializable;
\item language-specific compilation schemes, creating implicit references to objects that are not serializable;
\item transitive references that inadvertently hold on to excessively large object graphs, creating memory leaks;
\item capturing references to mutable objects, leading to race conditions in a concurrent setting;
\item unknowingly accessing unstable object members such as method calls, which in a distributed setting can have logically different meanings on different machines;
\end{itemize}

Given all of these issues, exposing functions in public APIs is a source of
headaches for authors of concurrent or distributed frameworks. Framework users
who stumble across any of these issues are put in a position where it's
unclear whether or not the encountered issue is a problem on the side of the
user or the framework, thus often adversely hitting the perceived reliability
of these frameworks and libraries. Furthermore, even internally-facing
function-based APIs pose serious software engineering challenges, since small
changes to a codebase can affect the environment captured by a given closure,
potentially rendering them suddenly not serializable or adding inadvertent
race conditions, thus making refactoring error-prone, and debugging difficult.

We argue that solving these problems in a principled way could lead to more
confidence on behalf of library authors in exposing functions in APIs, leading
to a potentially wide array of new frameworks.
% , such as for concurrent and
% distributed functional reactive programming.

This paper takes a step towards more principled {\em function-passing style}
by introducing a new closure-like abstraction, {\em spores}, which avoids
typical hazards when using closures in a concurrent or distributed setting
through controlled variable capture and constraints for captured types. Using
type-based constraints, spores allow expressing a variety of ``safe''
closures, such as guaranteed-serializable closures, closures which capture
only immutable types, or closures with customizable user-defined type-based
properties. Finally, we argue that by principle of a type-based approach,
spores have the added benefit of being able to benefit from optimization,
further safety via type system extensions, and verification opportunities.

The design of spores is guided by the following principles:

\begin{itemize}
\item {\bf Lightweight Design} to be practical for inclusion in the
full-featured Scala language. To enable a robust integration with the host
language, existing type system features are reused instead of extended.

\item {\bf Supporting existing practice}. When using frameworks for distributed
programming like Spark, closures should follow certain conventions with
respect to captured variables to avoid hazards. Spores enforce
such conventions at compile-time, where previously there was no tool support.

\item {\bf Lightweight syntactical footprint}. On the one hand, spores are designed to
make working with closures safer by making variable capture explicit. On the
other hand, implicit conversion between regular functions and spores enable
the use of spores also in existing closure-heavy code.

\item {\bf Type-based constraints} enable libraries and frameworks to restrict the types
that are captured by spores. For example, this allows enforcing that certain
spores only capture immutable objects. Surprisingly, we found that Scala's
type system let's us express a variety of constraints using existing type-
systematic features such as type refinements and implicits.

\item {\bf Custom, user-defined, type-based closure semantics}. Spores should enable customizing the semantics of variable capture based on user-defined types. It should be possible to use existing type-based mechanisms to express a variety of user-defined requirements for captured types.

\item {\bf Composability}. Spores, regardless of their constraints, should be logically and intuitively composable, while preserving their semantic properties.

\item {\bf More reliable and trustworthy APIs for library authors}. Spores enable library
authors to confidently release libraries that expose functions in the user-
facing API without concern of runtime exceptions.

\end{itemize}

\subsection{Selected Related Work}

Cloud Haskell~\cite{CloudHaskell} provides statically guaranteed-serializable
closures by either rejecting environments outright, or by allowing manual
capturing, requiring the user to explicitly specify and pre-serialize the
environment in combination with top-level functions (enforced using a new
\verb|Static| type constructor). That is, in Cloud Haskell, to create a
serializable closure, one must explicitly pass the serialized environment as a
parameter to the function -- this requires users to have to completely
reorganize any closure they wish to be made serializable. In contrast, spores
do not require users to manually factor out, manage, and serialize their
environment; spores require only that {\em what} is captured is specified, not
{\em how}. Furthermore, spores are more general than Cloud Haskell's
serializable closures -- user-defined type constraints enable spores to
express more properties than just serializability, like thread-safety,
immutability, or any other user-defined property. In addition, spores allow
restricting captured types in a way that's integrated with object-oriented
concerns, such as subtyping and open class hierarchies.

In industry, a number of languages have adopted or proposed pragmatic
solutions to allow control over the environment of closures.

C++11~\cite{CplusplusLambas} has introduced syntactic rules for explicit
capture specifications that indicate which variables are captured and how (by
reference, or by making a copy). Since the capturing semantics is purely
syntactic, a capture specification is only enforced at closure creation time.
Thus, when composing two closures, the capture semantics is not preserved.
Spores, on the other hand, capture such specifications at the level of types,
enabling composability. Furthermore, spores' type constraints enable more
general type-directed control over capturing than capture-by-value or
capture-by-reference alone.

There is a preliminary proposal for closures in the Rust
language~\cite{RustFunctions} that allows describing the closed-over variables
in the environment using closure bounds, requiring captured types to implement
certain traits. These closure bounds are limited to a small set of built-in
traits to enforce properties like sendability. Spores on the other hand enable
user-defined property definition, allowing for greater customizability of
closure capturing semantics. Furthermore, in Rust, the environment of a
closure must always be allocated on the stack (although not necessarily the
top-most stack frame). Spores, like regular Scala closures, do not have such a
restriction.

{Java 8}~\cite{JavaLambdas,JavaLambdaTranslation} introduces a limited type of
closure -- that is, closures are only permitted to capture variables that are
effectively-final. Like with Scala's standard closures, variable capture is
implicit which can lead to accidental captures that spores are designed to
avoid. Although serializability can be requested at the level of the type
system using newly-introduced intersection types in \mbox{Java 8}, there is no
guarantee about the absence of runtime exceptions as there is for spores.
Finally, spores additionally allow specifying type-based constraints for
captured variables that are more general than serializability alone.

We discuss other related work in Section~\ref{sec:related-work}.

\subsection{Contributions}

This paper makes the following contributions:

\begin{itemize}
\item We introduce a closure-like abstraction and type system, called ``spores'', which avoids typical hazards when using closures in a concurrent or distributed setting through controlled variable capture and customizable user-defined constraints for captured types.

\item We introduce an approach for type-based constraints that allow expressing a variety of properties from the literature including, but not limited to, serializability, immutability, and thread-safety.

\item We present a formalization of spores with type constraints and prove soundness of the type system.

\item We present an implementation of spores in and for the full Scala language\footnote{\sporeurl}.

\item We demonstrate the practicality through an small empirical study using a collection of Scala programs, and show the power of these guarantees through a case study of new distributed and concurrent frameworks that this safe foundation for migratable closures enables.

\end{itemize}

% With the growing trend towards big data, cloud computing and mobile
% applications, distributed programming has entered the mainstream. The
% traditional view of software development as being focused on a program running
% on a single machine, interacting directly with the user, has become largely
% obsolete. Popular paradigms in software engineering such as software as a
% service (SaaS), RESTful services, or the rise of multitudes of different
% models and systems for big data processing and interactive analytics, evidence
% this trend. Whether we consider a cluster of hundreds of commodity machines
% churning through a massive data-parallel job, or a smartphone interacting with
% a social network, all are ``distributed'' jobs, and all share the need to
% interact in typically asynchronous, reactive ways with other clients or
% services.

% Meanwhile, concurrently, functional programming has been undeniably gaining
% traction in recent years, as is evidenced by the ongoing trend of
% traditionally object-oriented or imperative languages being extended with
% functional features, such as lambdas in {Java 8}~\cite{JavaLambdas},
% C++11~\cite{CplusplusLambas}, and Visual Basic 9~\cite{Meijer}, the perceived
% importance of functional programming in general empirical studies on software
% developers~\cite{PLAdoption}, and the measurable popularity of functional
% programming massively online open courses (MOOCs)~\cite{ICSEMOOC}.

% One reason for the rise in popularity of functional programming languages and
% features within object-oriented communities is the basic philosophy of
% transforming immutable data via function application, and the observation that
% this mode of reasoning about data in parallel, concurrent, and distributed
% code is more tractable.\todo{I'd like to cite anything here, even a SO
% article}~{Java 8}'s monadic-style optionally parallel
% collections~\cite{JavaLambdas}, Scala's parallel
% collections~\cite{ScalaParColls} and concurrent dataflow~\cite{FlowPools}
% collections are just some efforts that evidence this trend.

% In particular, data parallelism is on the rise...?

% Data-parallelism has long been a big deal for GPU programming, distribution,
% etc.  The central idea behind data parallelism is to distribute data across
% different workers (whether they be threads or nodes in a cluster), and to
% execute a single set of instructions (SIMD) on that data.

% Data-parallelism -> functional, natural~\cite{Nova}
% FP for parallelism, data-parallel~\cite{Eden, DataParallelHaskell} (Eden is also distributed?)

% that data parallelism is important
% both in a multi-core (non-dist.) and a distributed setting
% in both cases, programming models for wide-spread langs make increasing use of lambdas
% in the multi-core category we have Scala's parcolls and Java 8 streams and DPJ (I think)
% in the dist. setting we have Spark as well as functional wrappers on top of Hadoop
% while data-parallel programming is very effective already in a purely functional setting (data-parallel haskell), it is still difficult in imperative OO langs with functional features
% due to hazards involving closures passed to data-parallel ops
% (they can capture mutable variables or data structures)

% where you said that we ship functionality to data in various systems
% and now we are using closures more and more for that
% like, languages are adopting closures and with it come more and more functional APIs

% we're moving in a directioxn where we ship closures where (a) data to-be-processed and required services are passed as arguments to closures and (b) data that needs to be shipped is captured and serialized
% % we're moving in a direction where we ship closures where (a) data to-be-processed is passed as arguments to closures
% % and (b) data that needs to be shipped is captured and serialized
% and  (c) services local to a node are either passed in as additional arguments by the dist. framework, or looked up from within closures
% % (c) services local to a node are looked up from within closures
% that's the pattern of ltu
% for Spark, an example for (c) is the SparkContext
% it's the main entry point to functionality provided by Spark
% but it's not serializable
% every node has its own singleton object
% so, I mean I think we don't have to say much more than that

% we could say that even though this characterization of data/services helps structuring dist. systems
% it's still brittle in practice because closures don't provide enough of a safety net
% thus, even though this pattern exists, it's not really used for structuring systems
% it's still brittle in practice because closures don't provide enough of a safety net
% thus, even though this pattern exists, it's not really used for structuring systems
% its main purpose is reducing boilerplate in big data/analytics applications

% FP touted to be enabler for parallel programming. Why?
% Models like Spark~\cite{Spark} and MapReduce~\cite{MapReduce} represent a paradigm shift in distributed
% computing. Here, the foundation is to move the functionality to the data. The
% foundational idea is to keep the data stationary, and to send functions to the
% data.

% \begin{fquote}[Jeff Epstein, et al.][Towards Haskell in the Cloud][2011]Yet it is not uncommon for some objects to essentially encode nothing more than a function. In a typical use case, a Java API defines an interface, informally called a ``callback interface," and expects a user to provide an instance of the interface when invoking the API. [...]
% \indent Many useful libraries rely on this pattern. It is particularly important for parallel APIs, in which the code to execute must be expressed independently of the thread in which it will run. The parallel-programming domain is of special interest, because as CPU makers focus their efforts on improving performance through a proliferation of cores, serial APIs are limited to a shrinking fraction of available processing power.\\
% \indent Given the increasing relevance of callbacks and other functional-style idioms, it is important that modeling code as data in Java be as lightweight as possible.
% \end{fquote}

% \begin{fquote}[Jeff Epstein, et al.][Towards Haskell in the Cloud][2011]Moreover, pure functions are idempotent; this means that functions running on failing hardware can be restarted elsewhere without the need for distributed transactions or other mechanisms for “undoing” effects.
%  \end{fquote}

% Using types, we can statically prevent much of the headaches that occur in dynamically typed languages when it comes to sending closures over the network.

% \begin{fquote}[Jeff Epstein, et al.][Towards Haskell in the Cloud][2011]In a distributed memory system, the most significant cost, in both energy
% and time, is data movement
%  \end{fquote}

% % \epigraph{In a distributed memory system, the most significant cost, in both energy
% % and time, is data movement.}{Simon Peyton Jones}
% % - from the Cloud Haskell Paper

% Functions don't concern only functional programming languages anymore.

% \todo{cite one or some of the lambda the ultimate papers here?}
% \todo{need to cite a few papers that claim that FP is the way forward for parallel and distributed computing}

% While purported to the way forward for bringing parallel, concurrent, and
% distributed programming to the mainstream, functional programming languages
% haven't yet taken off as the languages of choice for distributed and
% concurrent programming in practice. For that, object-oriented languages with functional features such as Scala or Java take the lead.

% Java 8's streaming APIs focus on concurrency, expose functions to the user.
% In Scala we have already gathered experience with these issues and thus we
% believe they will apply to Java 8 as well

% For both OO and functional languages, there still exist numerous hurdles at
% the language-level for even these most basic functional building blocks to
% overcome in order to be reliable and easy to reason about in concurrent or
% distributed environments. For instance, a natural model for
% functional languages to support is that of {\em moving functionality to
% data}\todo{this is a really common idea, cite a bunch of stuff that aims for
% this}. While a popular idea, there exist few distributed systems which embody
% this approach, the most notable of which being Spark~\cite{Spark}, a fault-tolerant,
% in-memory distributed collections abstraction.

% \todo{talk about issues of serializing lambdas in Java, Scala -- OO languages. Then discuss the problems of functional languages like Haskell too}

% This paper takes a step towards more principled {\em function-passing
% style}\todo{should probably say something like ``open programming''}~by
% introducing a new closure-like abstraction and type system that can guarantee
% closures to be serializable, thread-safe, or even have user-defined
% properties, called {\em spores}. Like normal functions, spores are composable

% We first present a and then focus on an implementation of spores in Scala, a
% hybrid object-oriented and functional programming language which faces the
% issues brought by closures on all sides -- the issues encountered in
% functional languages such as Haskell~\cite{CloudHaskell}, as well as issues
% encountered by integrating closures with an object system, inheritance and
% subtyping.

% The design of spores is guided by the following principles:
% \begin{itemize}
% \item {\bf A lightweight design} to be practical for inclusion in the
% full-featured Scala language. To enable a robust integration with the host
% language, existing type system features are reused instead of extended.

% \item {\bf Supporting existing practice}. When using frameworks for distributed
% programming like Spark, closures should follow certain conventions with
% respect to captured variables to avoid hazards. In some cases spores enforce
% such conventions at compile-time, where previously there was no tool support.

% \item {\bf Lightweight syntactical footprint}. On the one hand, spores are designed to
% make working with closures safer by making variable capture explicit. On the
% other hand, implicit conversion between regular functions and spores enable
% the use of spores also in closure-heavy code.

% \item {\bf Type-based constraints} enable libraries and frameworks to restrict the types
% that are captured by spores. For example, this allows enforcing that certain
% spores only capture immutable objects. Surprisingly, we found that Scala's
% type system let's us express a variety of constraints using existing type-
% systematic features such as type refinements and implicits.

% \item {\bf Serialization not baked in}, can focus on and attach other properties
% like thread-safety or mutability.

% \item {\bf More reliable/trustworthy APIs for library authors}. Enables library
% authors to confidently release libraries that expose functions in the user-
% facing API without concern of runtime exceptions.
% \end{itemize}

% One reason that this model of distributed computing has not pre- viously been
% brought to Haskell is that it requires a way of running code on a remote
% system. Our work provides this, in the form of a novel method for serializing
% function closures.
% - Haskell in the Cloud paper

% Cloud Haskell~\cite{CloudHaskell} also makes an effort to provide statically-guaranteeable serializable closures. They are limited to serializability, are
% limited to top-level functions, and reject an environment or allow controlled
% capturing, preserialization of the environment.  For us, management of the
% environment is automatic, we don't need to preserialize the env, and our type
% constraints are very different. We provide a solution for imperative OO languages
% as well. Their stuff is mostly top-level, whereas in Scala, closures usually refer
% to local objects nested within other objects

% == NOTES ==
% We introduce a model for concurrent and distributed computing where the data is stationary and functions are mobile. This is a dual to the actor model, where functionality is stationary, and data is made mobile.

% In theory you can migrate actors, in practice, you cannot. Can we use these insights to find a way to migrate actors cleanly?

% We describe a pattern for distributing lambdas in a principled way. We
% implement this pattern in the new spore-agent abstraction which enforces the
% rules of the pattern.

% We discuss problems in distributed programming that are simplified using the
% spore-agent abstraction compared to equivalent actor-based solutions.
% Moreover, we argue that spore-agents are dual to actors.

% Finally, we show that this spore-agent abstraction can be implemented
% efficiently; we show that our spore-agent implementation can outperform state-
% of-the-art actor frameworks on a set of applications where distributing
% lambdas is at the core.

% Distributed programming supposed to be made easier by going functional. But a problem of object-oriented languages with lambdas and functional programming languages alike is that functions can have free variables. Closures that close over some tricky environment. This means that regardless of the paradigm, framework designers can't confidently expose lambdas in APIs to users.

% Call on trend towards in-memory distributed computing.

% The design of our framework is guided by the following principles...

% In this paper, we present a model and type system for introducing general {\em type constraints}, and demonstrate its applicability


% \subsection{Problems of actors}

% \begin{itemize}

% \item Migrating actors is not supported in mainstream actor frameworks and
% languages. For example, neither Akka nor Erlang support migrating actors from
% one node to another.

% \item Sending spores as messages to actors comes with the problem of matching
% on the types of the spore which is not supported due to erasure in Scala.

% \end{itemize}

% We describe a pattern for distributing lambdas in a principled way. We
% implement this pattern in a family of spore-agent abstractions which enforce
% the rules of the pattern.

% {\bf Motivations:}
% \begin{itemize}
% \item problems with distribution
% \item problems with concurrency
% \end{itemize}

% \subsection{Contributions}

% \begin{itemize}
% \item a description, general model, and type system of \textit{spores}, functions that are guaranteed to be
% \item a full soundness proof
% \item a general system for applying type-constraints to arbitrary properties beyond serializability or thread-safety.
% \item an implementation for Scala
% \item a demonstration of the practicality of our approach via a small
% empirical study, as well as a case analysis of new distributed and concurrent
% frameworks that this safe  foundation for migratable closures enables.

% \end{itemize}

% \subsection{Brainstorming}

% Should focus on re-usable building blocks. Spores are re-usable by several
% frameworks. What else is re-usable? A composition mechanism?

% One thing could be to provide a mechanism to compose spores, but leave it open
% how to execute that on a distributed middleware.

% Make it easy to build an execution engine like the one from Spark.


% This paper makes the following contributions:

% \subsection{Jargon?}

% Section on jargon. Lambdas, functions, closures. Introduce a baseline
% vocabulary with respect to environment for duration of paper.

% \subsection{Functions Across Languages}

% A survey of the way different languages deal with functions and closures. How
% the environment is kept, scoping, stack frames, etc.

% Languages to compare with closures:

% \begin{itemize}
% \item Scala
% \item Objective C
% \item Java
% \item VB.net
% \item Python
% \item Ruby
% \item Javascript
% \item Lisp
% \item Perl
% \item Lisp
% \item Racket
% \item Scheme
% \item Clojure
% \item Rust
% \item Go
% \item Dart
% \item Haskell
% \item ML
% \item C++
% \item Lua
% \item Smalltalk
% \item ECMAScript
% \item C\# (Supports functions, but not closures. Emulates with delegates.)
% \end{itemize}

% \section{Lambda the Ultimate Distributive}

% Lambdas provide a simple, principled way of handling the different kinds of
% objects that are fundamental in distributed systems. This section describes a
% pattern that we call ``Lambda the Ultimate Distributive'' for working with
% lambdas in a distributed environment.

% In a distributed system, different kinds of objects have to be handled in
% different ways. Some of these objects are data that should be processed (like
% a large collection), some of these objects represent services of the runtime
% environment (e.g., for scheduling or communication).

% Given the kind of object that is involved in a distributed computation, the
% object has to be handled specially:
% \begin{itemize}

% \item (a) an object can be data that must be shipped between nodes to accomplish
% some task;

% \item (b) an object can be a runtime service, such as a scheduler, which is not
% shippable, but must be used on each node;

% \item (c) an object can be data which is not shippable for some reason, but might
% have to be retrieved on multiple nodes.

% \end{itemize}

% The idea of ``lambda the ultimate distributive'' is that closures can handle
% all of these kinds of objects in a principled and well-defined way (no
% guessing needed):

% \begin{itemize}

% \item objects of kind (a) can be safely captured by the closure since they are shippable;

% \item objects of kind (b) must be parameters of the closure; on each machine,
% a reference the (local) runtime environment is passed as an argument to the
% closure; (Individual services, like a scheduler, could be fields of a
% wrapping ``environment'' object.)

% \item objects of kind (c) must be parameters of the closure; on each machine,
% they have to be retrieved first using an object of kind (b), and then passed
% to the closure as an argument.

% \end{itemize}

% Using lambdas in this way allows modeling most situations \todo{make less vague}
% in a distributed system in a principled way, without special tricks or
% hacks. However, correctly using lambdas according to the above pattern
% requires discipline, since properties such as serializability of captured
% objects are typically not enforced by the language or type system.

% \subsection{Limitations of simple lambdas}

% Here we can point out problems of regular lambdas: accidentally capturing
% references to enclosing objects, accidentally capturing non-serializable
% objects, incorrect use of lambdas in concurrent code (one motivation for
% parallel closures), accidentally accessing unstable members of captured
% objects (important in particular in languages supporting the uniform access
% principle like Eiffel [check], Smalltalk [check], and Scala).


\section{Spores}

Spores are a closure-like abstraction which aim to give users [state goals nicely].
They achieve this by (a) enforcing a specific syntactic shape which, and (b) adding a type system.
We describe below the syntactic shape of spores, and in Section~\ref{sec:type-constraints} we informally describe the type system.

in a later section~\ref{}, we'll describe how to use spores with the type system extension proposed in this paper.


A spore is a closure with a specific shape that dictates how the environment of a spore is declared. In general, a spore has the following shape:
\begin{lstlisting}
spore {
  val y1: S1 = <expr1>
  ...
  val yn: Sn = <exprn>
  (x: T) => {
    // ...
  }
}
\end{lstlisting}
A spore consists of two parts: the header and the body. The list of value definitions at the beginning is called the spore header. The header is followed by a regular closure, the spore's body. The characteristic property of a spore is that the body of its closure is only allowed to access its parameter, values in the spore header, as well as top-level singleton objects (public, global state). In particular, the spore closure is not allowed to capture variables in the environment. Only an expression on the right-hand side of a value definition in the spore header is allowed to capture variables.

By enforcing this shape, the environment of a spore is always declared explicitly in the spore header which avoids accidentally capturing problematic references. Moreover, and that's important for OO languages, it's no longer possible to accidentally capture the \verb|this| reference.

Note that the evaluation semantics of a spore is equivalent to a closure obtained by leaving out the \verb|spore| marker:

\begin{lstlisting}
{
  val y1: S1 = <expr1>
  ...
  val yn: Sn = <exprn>
  (x: T) => {
    // ...
  }
}
\end{lstlisting}

In Scala, the above block first initializes all value definitions in order and then evaluates to a closure that captures the introduced local variables y1, ..., yn. The corresponding spore has the exact same evaluation semantics. What's interesting is that this closure shape is already used in production systems such as Spark to avoid problems with accidentally captured \verb|this| references. However, in these systems the above shape is not enforced, whereas with spores it is.


\subsubsection{Spore type}

The result type of the \verb|spore| constructor is not a regular function type, but a subtype of one of Scala's function types. This is possible, because in Scala functions are instances of classes that mix in one of the function traits. For example, the trait for functions of arity one looks like this:
\footnote{For simplicity we omit definitions of the \verb+andThen+ and \verb+compose+ methods in the definition of \verb+Function1+.}

\begin{lstlisting}
    trait Function1[-A, +B] {
      def apply(x: A):  B
    }
\end{lstlisting}

The \verb|apply| method is abstract; a concrete implementation applies the body of the function that's being defined to the argument \verb|x|. Functions are contravariant in their argument type A, indicated using the ``\verb|-|'' symbol, and covariant in their result type B, indicated using the ``\verb|+|'' symbol.

The type of a spore of arity one is a subtype of \verb|Function1|:

\begin{lstlisting}
    trait Spore[-A, +B] extends Function1[A, B] {
      type Captured
      type Excluded
    }
\end{lstlisting}

The \verb|Spore| trait has two abstract type members. In a concrete spore the \verb|Captured| type member is defined to be a tuple type with the types of all captured variables. We will talk about the Excluded type member in a later section.

For example, the spore

\begin{lstlisting}
    spore { val y1: String = expr1; val y2: Int = expr2; (x: Int) => y1+y2+x }
\end{lstlisting}

would have the refinement type

\begin{lstlisting}
    Spore[Int, String] {
      type Captured = (String, Int)
    }
\end{lstlisting}

(We omit the Excluded type member for simplicity; it is explained in detailed below.)

Using the `Spore` trait methods can require argument closures to be spores:

\begin{lstlisting}
    def sendOverWire(s: Spore[Int, Int]): Unit = ...
\end{lstlisting}

This way, libraries and frameworks can enforce the use of spores instead of plain closures, thereby reducing the risk for common programming errors (see Section~\ref{case studies}). Also note that in the above case, the \verb|Captured| (and \verb|Excluded|) type member is not specified, meaning it is left abstract; this way a spore type is compatible with the above parameter type regardless of which types are captured as long as the parameter and result types match.


By representing the environment of spores using refinement types, it is possible to preserve the constraints of spores when they are composed.

For example, assume we are given two spores s1 and s2 with types

\begin{lstlisting}
    Spore[Int, String] { type Captured = (String, Int) }
\end{lstlisting}

and

\begin{lstlisting}
    Spore[String, Int] { type Captured = Nothing }
\end{lstlisting}

respectively. The fact that the Captured type in the second spore is defined to be \verb|Nothing| means that the spore does not capture anything (\verb|Nothing| is Scala's bottom type). The composition of s1 and s2, written s1 compose s2, would have the refinement type

\begin{lstlisting}
    Spore[String, String] { type Captured = (String, Int) }
\end{lstlisting}

Note that the Captured type member of the result spore is equal to the Captured type of s1, since it is guaranteed that the result spore does not capture more than what s1 already captures. As we can see, not only are spores composable, but their also (refinement) types. In Section~\ref{} we discuss the two kinds of type constraints that our system provides, namely, excluded types and context bounds for captured types.


\subsection{Basic Usage}

simple spore (concrete example)
for-comprehensions (example using that guy above)
composition

\subsection{Advanced Usage and Type Constraints}

two kinds, excluded types which achieve [fill in] and context bounds for captured types which achieve [fill in]

\subsubsection{Excluded Types}

Libraries and frameworks for concurrent and distributed programming, such as Akka and Spark, typically have requirements to avoid capturing certain types in closures that are used together with library-provided objects and methods. For example, when using Akka, one should not capture variables of type `Actor`; in Spark, one should not capture variables of type `SparkContext`.

Such restrictions can be expressed in our system by excluding types from being captured by spores, using refinements of the Spore trait. For example, the following refinement type forbids capturing variables of type Actor:

\begin{lstlisting}
type SporeNoActor[-A, +B] = Spore[A, B] {
  type Excluded <: No[Actor]
}
\end{lstlisting}
\noindent
Note the use of the auxiliary type constructor No: it enables the exclusion of multiple types while supporting desired sub-typing relationships. Its definition is very simple:

\begin{lstlisting}
trait No[-T]
\end{lstlisting}
\noindent
(We explain the "-" annotation on the type parameter below when discussing sub-typing.) For example, exclusion of multiple types can be expressed as follows:

\begin{lstlisting}
type SafeSpore = Spore[Int, String] {
  type Excluded = No[Actor] with No[Util]
}
\end{lstlisting}

Given Scala's sub-typing rules for refinement types, a spore refinement excluding a superset of types excluded by an "otherwise type-compatible" spore is a subtype. For example, SafeSpore is a sub type of SporeNoActor[Int, String].


\paragraph{Subtyping}

Using some frameworks typically user-defined subclasses are created that extend framework-provided types. However, the extended types are sometimes not safe to be captured. For example, in Akka, user-created closures should not capture variables of type \verb|Actor| and any subtypes thereof. To express such a constraint in our system we define the \verb|No| type constructor to be contravariant in its type parameter; this is the meaning of the "-" annotation in the type declaration \verb|trait No[-T]|.

As a result, the following refinement type is a supertype of type \verb|SporeNoActor[Int, Int]| defined above (we assume \verb|MyActor| is a subclass of \verb|Actor|):

\begin{lstlisting}
type MySpore = Spore[Int, Int] {
  type Excluded <: No[MyActor]
}
\end{lstlisting}
\noindent
It is important that \verb|MySpore| is a supertype and not a subtype of \verb|SporeNoActor[Int, Int]|, since an instance of \verb|MySpore| could capture some other subclass of \verb|Actor| which is not itself a subclass of \verb|MyActor|. Thus, it would not be safe to use an instance of \verb|MySpore| where an instance of \verb|SporeNoActor[Int, Int]| is required. On the other hand, an instance of \verb|SporeNoActor[Int, Int]| is safe to use in place of an instance of \verb|MySpore|, since it is guaranteed not to capture \verb|Actor| or any of its subclasses.


\subsubsection{Attaching Properties with Context Bounds}

The fact that for spores a certain shape is enforced is very useful. However, in some situations this is not enough. For example, using closures in a concurrent setting is very error-prone, because of the fact that it's possible to capture mutable objects which leads to race conditions. Thus, closures should only capture immutable objects to avoid interference. However, such constraints cannot be enforced using the spore shape alone (captured objects are stored in constant values in the spore header, but such a constant might still refer to a mutable object).

In this section we introduce a form of type-based constraints called "context bounds" that can be attached to a spore which enforce certain type-based properties for all captured variables of a spore.\footnote{The name ``context bound'' is used in Scala to refer to a particular kind of implicit parameter that is added automatically if a type parameter has declared such a context bound. Our proposal essentially adds context bounds to type members.}

Taking another example, it might be necessary for a spore to require the availability of instances of a certain type class for the types of all its captured variables. A typical example for such a type class is Pickler: types with an instance of the Pickler type class can be pickled using a new pickling framework for Scala [oopsla-pickling]. To be able to pickle a spore, it's necessary that all its captured types have an instance of Pickler.\footnote{A spore can be pickled by pickling its environment and the fully-qualified class name of its corresponding function class.}

Spores allow expressing such a requirement using implicit properties. The idea is that if there is an implicit of type Property[Pickler] in scope at the point where a spore is created, then it is enforced that all captured types in the spore header have an instance of the Pickler type class:

\begin{lstlisting}
import spores.withPickler

spore {
  val name: String = <expr1>
  val age: Int = <expr2>
  (x: String) => {
    // ...
  }
}
\end{lstlisting}

While an imported property does not have an impact on how a spore is constructed (besides the property import), it has an impact on the result type of the spore macro. In the above example, the result type would be a refinement of the Spore type:

\begin{lstlisting}
Spore[String, Int] {
  type Captured = (String, Int)
  val captured: Captured
  implicit val ev0 = implicitly[Pickler[(String, Int)]]
  (x: String) => {
    // ...
  }
}
\end{lstlisting}

The refinement type contains a type member Captured which is defined to be a tuple of all the captured types. The values of the actual captured variables are accessible using the captured value member. What's more, the refinement type contains for each type class that's required an implicit value with a type class instance for type Captured.

Such implicit values allow retrieving a type class instance for the captured types of a given spore using Scala's implicitly function as follows:

\begin{lstlisting}
    val s = spore { ... }
    implicitly[Pickler[s.Captured]]
\end{lstlisting}

Note that s.Captured is defined to be the type of the environment of spore s: a tuple with all types of captured variables.

\paragraph{Expressing context bounds at the method level}

Using the above types and implicits it's now possible for a method to require argument spores to have certain context bounds. For example, requiring argument spores to have picklers defined for their captured types can be done as follows:

\begin{lstlisting}
def m[A, B](s: Spore[A, B])(implicit p: Pickler[s.Captured]) = {
  // ...
}
\end{lstlisting}


\subsubsection{Defining Custom Properties}

Properties can be introduced using the Property trait (provided by the spores library):

\begin{lstlisting}
    trait Property[T]
\end{lstlisting}

A custom property such as for immutable types can be introduced using a generic trait and an implicit property object that mixes in the above \verb|Property| trait:

\begin{lstlisting}
object safe {
  trait Immutable[T]
  implicit object immutableProp extends Property[Immutable]
  ...
}
\end{lstlisting}
\noindent
The next step is to mark selected types as immutable by defining an implicit object extending the desired list of types, each type wrapped in the \verb|Immutable| type constructor:

\begin{lstlisting}
object safe {
  ...
  import scala.collection.immutable.{Map, Set, Seq}
  implicit object collections extends Immutable[Map[_, _]] with
    Immutable[Set[_]] with Immutable[Seq[_]] with ...
}
\end{lstlisting}
\noindent
The above definitions allow us to create spores that are guaranteed to capture only types \verb|T| for which an implicit of type \verb|Immutable[T]| exists.

It's also possible to define compound properties by mixing in multiple traits into an implicit property object:

\begin{verbatim}
    implicit object myProps extends Property[Pickler] with Property[Immutable]
\end{verbatim}

By making this compound property available in a scope within which spores are created (for example, using an import), it is enforced that those spores have both the context bound Pickler and the context bound Immutable.


\subsubsection{Composition}

Now that we've introduced type constraints in the form of excluded types and context bounds, we present generalized composition rules for the types of spores with such constraints.

To precisely describe the composition rules, we introduce the following notation: the function $Excluded$ returns for a given refinement type the set of types that are excluded; the function $Captured$ returns for a given refinement type the set of type that are captured. Using these two mathematical functions, we can precisely specify how the type members of the resulting spore refinement type are computed. (We use the syntax \verb|.type| to refer to the singleton types of the argument spores and the result, respectively.)

\begin{enumerate}

\item $Captured(res.type) = Captured(s1.type), Captured(s2.type)$

\item $Excluded(res.type) = \{ T \in Excluded(s1.type) \cup Excluded(s2.type) ~|~ T \notin Captured(s1.type) \cup Captured(s2.type) \}$

\end{enumerate}

The first rule expresses the fact that the sequence of captured types of the resulting refinement type is simply the concatenation of the captured types of the argument spores.

The second rule expresses the fact that the set of excluded types of the result refinement type is defined as the set of all types that are excluded by one of the argument spores, but that are not captured by any of the argument spores.

For example, assume two spores s1 and s2 with types
\begin{lstlisting}
Spore[Int, String] {
  type Captured = (Int, Util)
  type Excluded = No[Actor]
}
\end{lstlisting}
\noindent
and
\begin{lstlisting}
Spore[String, Int] {
  type Captured = (String, Int)
  type Excluded = No[Actor] with No[Util]
}
\end{lstlisting}
\noindent
The result of composing the two spores, \verb|s1 compose s2|, has thus the following result type:

\begin{lstlisting}
Spore[String, String] {
  type Captured = (Int, Util, String, Int)
  type Excluded = No[Actor]
}
\end{lstlisting}

\paragraph{Loosening constraints}
Also should show how to loosen constraints!!
Constraints can be loosened using the regular type widening rules.

Let's say we have a spore with the following elaborate refinement type:
\begin{lstlisting}
Spore[String, Int] {
  type Captured = (String, Int)
  type Excluded = No[Actor] with No[Util]
}
\end{lstlisting}

Then it's possible to soundly drop constraints by assigning the spore to a variable of a super type such as

\begin{lstlisting}
Spore[String, Int] {
  type Captured
  type Excluded <: No[Actor]
}
\end{lstlisting}

\begin{lstlisting}
def m(s: Spore[String, Int] { type Excluded <: No[Actor] })

s1: Spore[String, Int] {
  type Captured = (String, Int)
  type Excluded = No[Actor] with No[Util]
}
m(s1)
\end{lstlisting}



% =========== BEGINNING OF OLD SPORES SECTION ===========
% \section{Spores}

% Spores are a closure-like abstraction which aim to give users . They achieve this by (a) enforcing a
% specific syntactic shape which, and (b) adding a type system.

% We would like to constrain the environment of a spore to express properties
% such as:
% \begin{itemize}
% \item All captured objects must be serializable
% \item All captured objects must be thread-safe
% \item Captured objects must not be marked as unsafe for capturing
% \end{itemize}
% \noindent
% Such properties restrict the types of objects that a spore is allowed to
% capture. A naive way to achieve this would be to annotate certain types as
% ``not-safe-for-capturing''. However, this approach would be too inflexible,
% since some desirable properties are orthogonal to others. For example, a
% \verb|Promise| type~\cite{promise-paper} would be safe for capturing by a
% spore used in a concurrent setting, whereas it would not be safe for capturing
% by a spore that is intended to be serialized.

% Thus, rather than permitting or disallowing types to be captured, in our
% approach spores can express properties that the types of captured variables
% are required to have. For example, a spore might require all captured types to
% be thread-safe. Properties can also be composed, enabling spores to require
% their captured types to have multiple properties. For example, in addition to
% thread-safety, a spore may require its captured types to be serializable.
% (Such a spore would be well-suited for computations that should be executed
% concurrently on the same computer or on another computer depending on the
% current work load.)

% \subsection{The \texttt{Spore} trait}

% To express type constraints we leverage two features of Scala's type system.
% First, in Scala a function is an object of a type that extends one of the
% predefined function traits. Thus, it is possible to subclass the predefined
% function types. Second, in Scala classes and traits can be refined with value
% and type members. As we will see, type members are a natural way to express
% type constraints.

% The type of single-argument spores is defined as follows:

% \begin{lstlisting}
%     trait Spore[-A, +R] extends Function1[A, R] {
%       type Capture[T]
%     }
% \end{lstlisting}
% \noindent
% The \verb|Spore| trait extends the \verb|Function1| trait~\todo{variance annotations} which is defined as
% follows in Scala's standard library:

% \begin{lstlisting}
%     trait Function1[-A, +R] extends AnyRef {
%       def apply(v: A): R
%       def andThen[B](g: R => B): A => B
%       def compose[B](g: B => A): B => R
%     }
% \end{lstlisting}
% \noindent
% The above definition expresses the fact that functions are objects with an
% \verb|apply| method; function composition is supported using the
% \verb|andThen| and \verb|compose| methods.

% The \verb|Capture[T]| type member of the \verb|Spore| trait is used to express
% requirements of all captured types \verb|T| in the form of ``interfaces'' that
% the captured types \verb|T| have to implement. As we will see below, instead
% of using Java-style interfaces, we use a combination of traits and
% implicits~\cite{Oliveira2010} akin to generalized interfaces in
% JavaGI~\cite{WehrT11} or concepts in C++.

% For example, a spore from \verb|Int| to \verb|String| that requires all
% captured types to be serializable would have the following type:
% \begin{lstlisting}
%     Spore[Int, String] {
%       type Capture[T] = Pickler[T]
%     }
% \end{lstlisting}
% \noindent
% The type member \verb|Capture[T]| is defined such that for each captured type
% \verb|T| there must exist an instance of type \verb|Pickler[T]| (a ``pickler''
% that can serialize or pickle objects of type \verb|T|). Such instances are
% made available and looked up using Scala's {\em implicits} as shown below.

% Just like requiring all captured types to have a certain property, sometimes
% it's necessary to prevent types with a certain property from being captured.
% While in principle this could be supported by requiring a property that's not
% satisfied only by those types we wish to exclude from capturing, it would
% require declaring all types that are not excluded, which would lead to a lot
% of boilerplate. Instead, we introduce another type member \verb|NoCapture[T]|
% used to express properties that lead to rejection of a captured type:\todo{also discuss subtypes}

% \begin{lstlisting}
%     trait Spore[-A, +R] extends Function1[A, R] {
%       type Capture[T]
%       type NoCapture[T]
%     }
% \end{lstlisting}
% \noindent
% Given this way of introducing constraints it's even possible to require
% captured types to have multiple properties. Since properties are expressed
% using traits, properties can be combined using mix-in composition. For
% example, a spore requiring all captured types to be both serializable and
% thread-safe would look as follows:

% \begin{lstlisting}
%     Spore[Int, String] {
%       type Capture[T] = Pickler[T] with ThreadSafe[T]
%     }
% \end{lstlisting}
% \noindent
% The idea behind the property \verb|ThreadSafe| is that thread-safe types could
% provide dummy implementations of the interface \verb|ThreadSafe[T]|. Moreover,
% the \verb|with| keyword is used to create the mix-in composition of
% \verb|Pickler[T]| and \verb|ThreadSafe[T]|. This composition is guaranteed to
% be defined if we require all properties to be direct descendants of
% \verb|AnyRef|, the top type of all reference types in Scala.

% Spores without constraints have types of the following form:
% \begin{lstlisting}
%     Spore[A, R] {
%       type Capture[T] = NoConstraint[T]
%       type NoCapture[T] = NoConstraint[T]
%     }
% \end{lstlisting}

% \subsection{Library-specific constraints}

% One important goal of spores is to make the use of closure-based libraries
% safer. This is enabled through subtyping: instead of using regular function
% types, libraries can require the use of subtypes of the \verb|Spore| trait
% introduced above.

% For example, a concurrency library could declare certain types as not safe to
% be captured by closures. We can express this using a type class, say,
% \verb|Unsafe[T]|. For each type that should not be captured we introduce a
% type class instance. For example, the following implicit object introduces an
% instance of the \verb|Unsafe| type class for type \verb|Actor|:

% \begin{lstlisting}
%     implicit object unsafeActor extends Unsafe[Actor]
% \end{lstlisting}
% \noindent
% To ensure that closures never capture types that are declared in such a way as
% unsafe, the library could use the following type alias in place of the regular
% type for functions of arity one:

% \begin{lstlisting}
%     type SafeSpore[A, R] = Spore[A, R] {
%       type NoCapture[T] <: Unsafe[T]
%     }
% \end{lstlisting}
% \noindent
% The \verb|NoCapture| type member of the above \verb|Spore| subtype is refined
% with the {\em upper type bound} \verb|Unsafe[T]|. This ensures that only spore
% types whose \verb|NoCapture| type members mix in \verb|Unsafe| are compatible
% (of course, they could be even more restrictive).

% \subsection{Introducing custom constraints}

% \subsection{Composing spores}


% Type constraints for a newly defined spore can be introduced as follows:
% \begin{lstlisting}
%     spore {
%       val s: String = // ...
%       (p: Int) => {
%         (1 to p).map(x => s).mkString("-")
%       }
%     }.with[Pickler]
% \end{lstlisting}
% \noindent

% [explain how to exclude certain types]

% [ideally safety properties of types can be declared retro-actively.]

% [how do we motivate other, custom properties?]
%
% =========== END OF OLD SPORES SECTION ===========


% \section{Function-Passing Paradigm}

% Frameworks this new paradigm would enable. Relationship to NoSQL?

% \subsection{General Model}

% Goal of general model: to take this mess of closures and functions and trouble with environments that pains people across languages and propose an abstraction that gets rid of this confusion.

% Include in the general model typing rules that makes this abstraction guaranteeably shippable
% given some baseline simplified formalism.

% \subsection{Meat}

% Need to work out structure of this and an appropriate title for this section. Would include stuff like type system and type constraints for spores perhaps (or keep that in the general model section?). Composing spores with type constraints. Subtyping and type constraints.

% \subsubsection{OO}

% The question of -- I capture something of type MyActor
% and the spore forbids capturing type Actor, and MyActor <: Actor
% then what happens.

\section{Formalization}\label{sec:formal}

\begin{figure*}[ht!]
  \centering

  $\ba[t]{l@{\hspace{2mm}}l}
t ::=     x                                 & \mbox{variable}
\\
\gap ~|~  (x: T) \Rightarrow t              & \mbox{abstraction}
\\
\gap ~|~  t~t                               & \mbox{application}
\\
\gap ~|~  \texttt{let}~x = t~\texttt{in}~t  & \mbox{let binding}
\\
\gap ~|~  \{ \seq{l = t} \}                 & \mbox{record construction}
\\
\gap ~|~  t.l                               & \mbox{selection}
\\
\gap ~|~  \texttt{spore}~\{~\seq{x : T = t}~; \seq{pn} ; (x: T) \Rightarrow t~\}  & \mbox{spore}
\\
\gap ~|~  \texttt{import}~pn~\texttt{in}~t  & \mbox{property import}
\\
\gap ~|~  t~\texttt{compose}~t              & \mbox{spore composition}
\\
 & \\
v ::=     (x: T) \Rightarrow t              & \mbox{abstraction}
\\
\gap ~|~  \{ \seq{l = v} \}                 & \mbox{record value}
\\
\gap ~|~  \texttt{spore}~\{~\seq{x : T = v}~; \seq{pn} ; (x: T) \Rightarrow t~\}  & \mbox{spore value}
\\
 & \\
T ::=     T \Rightarrow T                   & \mbox{function type} \\
\gap ~|~  \{ \seq{l : T} \}                 & \mbox{record type}   \\
\gap ~|~  \mathcal{S}                       & \mbox{}
\\
\mathcal{S} ::= T \Rightarrow T~\{~\texttt{type}~\mathcal{C} = \seq{T}~;~\seq{pn}~\}   & \mbox{spore type}
\\
\gap ~|~  T \Rightarrow T~\{~\texttt{type}~\mathcal{C}~;~\seq{pn}~\}   & \mbox{abstract spore type}
\\
P \in pn \rightarrow \mathcal{T} & \mbox {property map}
\\
\mathcal{T} \in \mathcal{P}(T)   & \mbox{type family}
%::= \epsilon & \mbox{type family} \\
%\gap ~|~ T, \mathcal{T} & \mbox{non-empty}\\
\\
 & \\
\Gamma ::=  \seq{x : T}          & \mbox{type environment}
\\
\Delta ::=  \seq{pn}             & \mbox{property environment}
\\
\ea$

  \caption{Core language syntax}
  \label{fig:syntax}
\end{figure*}

We formalize spores in the context of a lambda calculus with records, shown in Figure~\ref{fig:syntax}. The language is in A-normal form~\cite{ANF} which means that all intermediate terms are named. Terms are standard except for the \texttt{spore} and the \texttt{import} terms. A \texttt{spore} term creates a new spore, and an \texttt{import} term imports a property name into the property environment within a lexically-scoped term; the property environment is used whenever a spore is created. This is explained in more detail below in Section~\ref{sec:typing}. The grammar of types is standard except for spore types. Spore types are refinements of function types. They additionally contain a sequence of captured types (which can be left abstract) and a sequence of property names.

\subsection{Subtyping}\label{sec:subtyping}

\begin{figure*}[ht!]
  \centering

\infrule[\textsc{S-Rec}]
{ \seq{l'} \subseteq \seq{l} \andalso l_i = l'_i \to T_i <: T'_i \land T'_i <: T_i
}
{ \{ \seq{l : T} \} <: \{ \seq{l' : T'} \}
}

\vspace{0.4cm}

\infrule[\textsc{S-Fun}]
{ T_2 <: T_1 \andalso R_1 <: R_2
}
{ T_1 \Rightarrow R_1 <: T_2 \Rightarrow R_2
}

\vspace{0.4cm}

\infrule[\textsc{S-Spore}]
{ T_2 <: T_1 \andalso R_1 <: R_2 \andalso \seq{pn'} \subseteq \seq{pn}
}
{ T_1 \Rightarrow R_1~\{~\texttt{type}~\mathcal{C} = \seq{S}~;~\seq{pn}~\} <: T_2 \Rightarrow R_2~\{~\texttt{type}~\mathcal{C} = \seq{S}~;~\seq{pn'}~\}
}

\vspace{0.4cm}

\infrule[\textsc{S-SporeAbs}]
{ T_2 <: T_1 \andalso R_1 <: R_2 \andalso \seq{pn'} \subseteq \seq{pn}
}
{ T_1 \Rightarrow R_1~\{~\texttt{type}~\mathcal{C} = \seq{S}~;~\seq{pn}~\} <: T_2 \Rightarrow R_2~\{~\texttt{type}~\mathcal{C}~;~\seq{pn'}~\}
}

\vspace{0.4cm}

\infrule[\textsc{S-SporeAbsAbs}]
{ T_2 <: T_1 \andalso R_1 <: R_2 \andalso \seq{pn'} \subseteq \seq{pn}
}
{ T_1 \Rightarrow R_1~\{~\texttt{type}~\mathcal{C}~;~\seq{pn}~\} <: T_2 \Rightarrow R_2~\{~\texttt{type}~\mathcal{C}~;~\seq{pn'}~\}
}

\vspace{0.4cm}

\infrule[\textsc{S-SporeFun}]
{
}
{ T_1 \Rightarrow R_1~\{~\texttt{type}~\mathcal{C} = \seq{S}~;~\seq{pn}~\} <: T_1 \Rightarrow R_1
}

\vspace{0.4cm}

\infrule[\textsc{S-SporeAbsFun}]
{
}
{ T_1 \Rightarrow R_1~\{~\texttt{type}~\mathcal{C}~;~\seq{pn}~\} <: T_1 \Rightarrow R_1
}

  \caption{Subtyping}
  \label{fig:subtyping}
\end{figure*}

The subtyping rule for spores is analogous to the subtyping rule for regular functions with respect to the argument and result types. Additionally, for two spore types to be in a subtyping relationship either their captured types have to be same or the supertype must have an abstract spore type. Also, it's possible that the subtype provides more properties $\seq{pn}$ than its supertype. The way both the captured type and the properties are modeled corresponds to the subtyping rule for refinement types in Scala.

\subsection{Typing rules}\label{sec:typing}

\begin{figure*}[ht!]
  \centering

\infrule[\textsc{T-Var}]
{ x : T \in \Gamma
}
{ \Gamma \vdash x : T
}

\vspace{0.4cm}

\infrule[\textsc{T-Sub}]
{ \Gamma \vdash t : T'  \quad  T' <: T
}
{ \Gamma \vdash t : T
}

\vspace{0.4cm}

\infrule[\textsc{T-Abs}]
{ \Gamma, x : T_1 \vdash t : T_2
}
{ \Gamma \vdash ((x: T_1) \Rightarrow t) : (T_1 \Rightarrow T_2)
}

\vspace{0.4cm}

\infrule[\textsc{T-App}]
{ \Gamma \vdash t_1 : T_1 \Rightarrow T_2 \quad
  \Gamma \vdash t_2 : T_1
}
{ \Gamma \vdash (t_1~t_2) : T_2
}

\vspace{0.4cm}

\infrule[\textsc{T-Let}]
{ \Gamma \vdash t_1 : T_1 \quad \Gamma, x : T_1 \vdash t_2 : T_2
}
{ \Gamma \vdash \texttt{let}~x = t_1~\texttt{in}~t_2 : T_2
}

\vspace{0.4cm}

\infrule[\textsc{T-Rec}]
{ \Gamma \vdash \seq{t : T}
}
{ \Gamma \vdash \{ \seq{l = t} \} : \{ \seq{l : T} \}
}

\vspace{0.4cm}

\infrule[\textsc{T-Sel}]
{ \Gamma \vdash t : \{ \seq{l : T} \}
}
{ \Gamma \vdash t.l_i : T_i
}

\vspace{0.4cm}

\infrule[\textsc{T-Import}]
{ \Gamma ; \Delta, pn \vdash t : T
}
{ \Gamma ; \Delta \vdash \texttt{import}~pn~\texttt{in}~t : T
}

\vspace{0.4cm}

\infrule[\textsc{T-Spore}]
{ \forall t_i \in \seq{t}.~\Gamma ; \Delta \vdash t_i : S_i \\
  \seq{x : S}, x : T_1 ; \Delta \vdash t : T_2 \\
  \forall pn \in \Delta.~\seq{S} \subseteq P(pn)
}
{ \Gamma ; \Delta \vdash \texttt{spore}~\{~\seq{x : S = t}~; (x: T_1) \Rightarrow t~\} : \\
  T_1 \Rightarrow T_2~\{~\texttt{type}~\mathcal{C} = \seq{S}~;~\Delta~\}
}

\vspace{0.4cm}

\infrule[\textsc{T-Comp}]
{ \Gamma \vdash t_1 : T_1 \Rightarrow T_2~\{~\texttt{type}~\mathcal{C} = \seq{S}~;~\Delta_1~\} \\
  \Gamma \vdash t_2 : U_1 \Rightarrow T_1~\{~\texttt{type}~\mathcal{C} = \seq{R}~;~\Delta_2~\} \\
  \Delta = \{ pn \in \Delta_1 \cup \Delta_2 ~|~ \seq{S} \subseteq P(pn) \land \seq{R} \subseteq P(pn) \}
}
{ \Gamma \vdash t_1~\texttt{compose}~t_2 : U_1 \Rightarrow T_2~\{~\texttt{type}~\mathcal{C} = \seq{S}, \seq{R}~;~\Delta~\}
}

  \caption{Typing rules}
  \label{fig:typing-rules}
\end{figure*}

When type-checking we assume the existence of a global property mapping $P$ from property names $pn$ to type families $\mathcal{T}$, much like a global class table in a core language like FJ~\cite{FJ}.

Operationally, spore composition is the same as function composition. However, type-wise there is a difference given the type refinements of spores. The compose rule (\textsc{T-Comp}) will always try to return a spore type with stronger properties (if possible); it does that by making use of the captured types and will add a property if indeed all captured types satisfy it. On the other hand, it's always possible to weaken the properties of a spore, by assigning a spore to a variable which has only weaker properties. This is enabled through spore subtyping and subsumption (\textsc{T-Sub}).

Note that there is no rule for spore application. The reason is that there is a subtyping relationship between spores and functions. Thus, according to rule (\textsc{T-Sub}) and the subtyping rules in Figure~\ref{fig:subtyping} we can already deduce that a spore has a function type; thus, the regular rule (\textsc{T-App}) applies.

\subsection{Operational semantics}\label{sec:opsem}
\begin{figure*}[ht!]
  \centering
\infrule[\textsc{E-App1}]
{ t_1 \rightarrow t'_1
}
{ t_1 t_2 \rightarrow t'_1 t_2
}

\vspace{0.4cm}

\infrule[\textsc{E-App2}]
{ t_2 \rightarrow t'_2
}
{ v_1 t_2 \rightarrow v_1 t'_2
}

\vspace{0.4cm}

\infax[\textsc{E-AppAbs}]
{ ((x : T) \Rightarrow  t) v \rightarrow [x \mapsto v]t
}

\vspace{0.4cm}

\infrule[\textsc{E-Let1}]
{ t_1 \rightarrow t'_1
}
{ \texttt{let}~x = t_1~\texttt{in}~t_2 \rightarrow \texttt{let}~x = t'_1~\texttt{in}~t_2
}

\vspace{0.4cm}

\infax[\textsc{E-Let2}]
{ \texttt{let}~x = v_1~\texttt{in}~t_2 \rightarrow [x \mapsto v_1]t_2
}

\vspace{0.4cm}

\infrule[\textsc{E-Rec}]
{ t_k \rightarrow t'_k
}
{ \{ \seq{l = v}, l_k = t_k, \seq{l' = t'} \} \rightarrow \{ \seq{l = v}, l_k = t'_k, \seq{l' = t'} \}
}

\vspace{0.4cm}

\infrule[\textsc{E-Spore}]
{ t_k \rightarrow t'_k
}
{ \texttt{spore}~\{~\seq{x : T = v}, x_k : T_k = t_k, \seq{x' : T' = t'}~; (x: T) \Rightarrow t~\} \rightarrow \\ \texttt{spore}~\{~\seq{x : T = v}, x_k : T_k = t'_k, \seq{x' : T' = t'}~; (x: T) \Rightarrow t~\}
}

\vspace{0.4cm}

\infrule[\textsc{E-AppSpore}]
{ \forall pn \in \seq{pn}.~\seq{T} \subseteq P(pn)
}
{ \texttt{spore}~\{~\seq{x:T=v};\seq{pn};(x':T)\Rightarrow t~\}v' \rightarrow \seq{[x \mapsto v]}[x' \mapsto v']t
}

\vspace{0.4cm}

\infrule[\textsc{E-Sel1}]
{ t \rightarrow t'
}
{ t.l \rightarrow t'.l
}

\vspace{0.4cm}

\infax[\textsc{E-Sel2}]
{ \{ \seq{l = v} \}.l_i \rightarrow v_i
}

\vspace{0.4cm}

\infax[\textsc{E-Imp}]
{ \texttt{import}~pn~\texttt{in}~t \rightarrow insert(pn, t)
}

\vspace{0.4cm}

\infrule[\textsc{E-Comp1}]
{ t_1 \rightarrow t_1'
}
{ t_1~\texttt{compose}~t_2\rightarrow t_1'~\texttt{compose}~t_2
}

\vspace{0.4cm}

\infrule[\textsc{E-Comp2}]
{ t_2 \rightarrow t_2'
}
{ v_1~\texttt{compose}~t_2\rightarrow v_1~\texttt{compose}~t_2'
}

\vspace{0.4cm}

\infrule[\textsc{E-Comp3}]
{ \Delta = \{ p~|~p \in \seq{pn},\seq{qn}.~\seq{T} \subseteq P(p) \land \seq{S} \subseteq P(p)\}
}
{ \texttt{spore}~\{~\seq{x:T=v};\seq{pn};(x':T')\Rightarrow t~\}~\texttt{compose}~\texttt{spore}~\{~\seq{y:S=w};\seq{qn};(y':S')\Rightarrow t'~\} \rightarrow \\ \texttt{spore}~\{~\seq{x:T=v}, \seq{y:S=w} ; \Delta ; (y': S') \Rightarrow \texttt{let}~z' = t'~\texttt{in}~[x' \mapsto z']t\}
}

  \caption{Operational Semantics}
  \label{fig:opsem}
\end{figure*}


\begin{figure*}[ht!]
  \centering

\infrule[\textsc{H-InsSpore1}]
{ \forall t_i \in \seq{t}.~insert(pn, t_i) = t'_i \\
  insert(pn, t) = t'
}
{ insert(pn, \texttt{spore}~\{~\seq{x:T=t};\seq{pn};(x':T)\Rightarrow t~\}) = \\ \texttt{spore}~\{~\seq{x:T=t'};\seq{pn}, pn;(x':T)\Rightarrow t'~\}
}

\vspace{0.4cm}

\infax[\textsc{H-InsSpore2}]
{ insert(pn, \texttt{spore}~\{~\seq{x:T=v};\seq{pn};(x':T)\Rightarrow t~\}) = \\
\texttt{spore}~\{~\seq{x:T=v};\seq{pn}, pn, pn;(x':T)\Rightarrow t~\}
}

\vspace{0.4cm}

\infax[\textsc{H-InsApp}]
{ insert(pn, t_1~t_2) = insert(pn, t_1)~insert(pn, t_2)
}

\vspace{0.4cm}

\infax[\textsc{H-InsSel}]
{ insert(pn, t.l) = insert(pn, t).l
}

  \caption{Helper functions}
  \label{fig:helper}
\end{figure*}


Q: what is the inversion lemma used for?

\begin{lemma}
\emph{(Inversion of the typing relation)}
\label{lem:inversion}
\begin{enumerate}
\item If $\Gamma \vdash x : T$ then $x : T \in \Gamma$.
\item If $\Gamma \vdash t_1 t_2 : R$ then there is some type $T_{11}$
\end{enumerate}
\end{lemma}


\begin{lemma}
\emph{(Canonical forms)}
\label{lem:canonical}
\begin{enumerate}

\item If $v$ is a value of type $\{ \seq{l : T} \}$, then $v$ is $\{ \seq{l = v} \}$ where $\seq{v}$ is a sequence of values.

\item If $v$ is a value of type $T \Rightarrow R$, then $v$ is either $(x: T) \Rightarrow t$ or \\ $\texttt{spore}~\{~\seq{x : T = v}~; \seq{pn} ; (x: T_1) \Rightarrow t~\}$ where $T <: T_1$ and $\seq{v}$ is a sequence of values.

% \item If $v$ is a value of type $T_1 \Rightarrow R_1~\{~\texttt{type}~\mathcal{C} = \seq{S}~;~\seq{pn}~\}$, then $v$ is \\ $\texttt{spore}~\{~\seq{x : S = v}~; \seq{pn} ; (x: T_1) \Rightarrow t~\}$ where $\seq{v}$ is a sequence of values.

% \item If $v$ is a value of type $T_1 \Rightarrow R_1~\{~\texttt{type}~\mathcal{C}~;~\seq{pn}~\}$, then $v$ is \\ $\texttt{spore}~\{~\seq{x : S = v}~; \seq{pn} ; (x: T_1) \Rightarrow t~\}$ where $\seq{v}$ is a sequence of values.

\end{enumerate}
\end{lemma}
\begin{proof}
According to the grammar in Figure~\ref{fig:syntax}, values in the core language can have three forms: $(x: T) \Rightarrow t$, $\{ \seq{l = v} \}$, and $\texttt{spore}~\{~\seq{x : T = v}~; \seq{pn} ; (x: T) \Rightarrow t~\}$ where $\seq{v}$ is a sequence of values.
For the first part, according to (\textsc{T-Rec}) and the subtyping rules, $v$ is $\{ \seq{l = v} \}$ where $\seq{v}$ is a sequence of values of types $\seq{T}$.
For the second part, according to the subtyping rules $v$ can have either type $T_1 \Rightarrow R_1$, $T_1 \Rightarrow R_1~\{~\texttt{type}~\mathcal{C} = \seq{S}~;~\seq{pn}~\}$, or $T_1 \Rightarrow R_1~\{~\texttt{type}~\mathcal{C}~;~\seq{pn}~\}$ where $T <: T_1$ and $R_1 <: R$. If $v$ has type $T_1 \Rightarrow R_1$, then according to the grammar and (\textsc{T-Abs}) $v$ must be $(x: T) \Rightarrow t$. If $v$ has either type $T_1 \Rightarrow R_1~\{~\texttt{type}~\mathcal{C} = \seq{S}~;~\seq{pn}~\}$ or type $T_1 \Rightarrow R_1~\{~\texttt{type}~\mathcal{C}~;~\seq{pn}~\}$, then according to the grammar and (\textsc{T-Spore}) $v$ must be $\texttt{spore}~\{~\seq{x : T = v}~; \seq{pn} ; (x: T_1) \Rightarrow t~\}$ where $\seq{v}$ is a sequence of values.
% Parts three and four are similar.
\end{proof}

Q: do we need to do a substitution lemma? Maybe because substitution within a spore is not obvious?

\begin{theorem}
\emph{(Progress)}
\label{th:progress}
Suppose $t$ is a closed, well-typed term (that is, $\vdash t : T$ for some $T$). Then either $t$ is a value or else there is some $t'$ with $t \rightarrow t'$.
\end{theorem}
\begin{proof}
By induction on a derivation of $t : T$. The only two interesting cases are the ones for application (where we might apply a spore to some argument), spores, and spore composition (TODO: verify).

Case \textsc{T-Spore}: $t = \texttt{spore}~\{~\seq{x : S = t}~; (x: T_1) \Rightarrow t~\}$, $\forall t_i \in \seq{t}.~\vdash t_i : S_i$, and $\seq{x : S}, x : T_1 \vdash t : T_2$. By the induction hypothesis, either all $\seq{t}$ are values, in which case $t$ is a value; or there is a term $t_i$ such that $t_i \rightarrow t_i'$ (since $\Gamma ; \Delta \vdash t_i : S_i$) and all $t_j$ where $j < i$ are values $v_j$. Thus, by (\textsc{E-Spore}), there is a $t'$ such that $t \rightarrow t'$.

Case \textsc{T-App}: $t = t_1~t_2$, $\vdash t_1 : T_1 \Rightarrow T_2$, and $\vdash t_2 : T_1$. By the induction hypothesis, either $t_1$ is a value $v_1$, or $t_1 -> t_1'$. In the latter case it follows from (\textsc{E-App1}) that there is a $t'$ such that $t \rightarrow t'$. In the former case, by the induction hypothesis $t_2$ is either a value $v_2$ or $t_2 \rightarrow t_2'$. In the former case by the canonical forms lemma we have that $v_2$ is either $(x: T) \Rightarrow t$ or $\texttt{spore}~\{~\seq{x : T = v}~; \seq{pn} ; (x: T_1) \Rightarrow t~\}$ where $T <: T_1$ and $\seq{v}$ is a sequence of values; thus, either (\textsc{E-AppAbs}) or (\textsc{E-AppSpore}) apply. In the latter case,

\end{proof}


\begin{theorem}
\emph{(Preservation)}
\label{th:pres}
If $\Gamma \vdash t : T$ and $t \rightarrow t'$, then $\Gamma \vdash t' : T$.
\end{theorem}


Goals:
\begin{itemize}
\item If an expression is type-correct, at runtime all its properties should be true. This is particularly
  interesting for spore expressions.
\item Furthermore, whenever an expression is reduced the result of the reduction should also be type-correct which
  makes sure it's static properties are preserved by reduction. (Preservation)
\item To achieve soundness of the type system, we would also like to show that if an expression in our core
  language is type-correct, then the expression can be reduced. (Progress)
\end{itemize}

Spores are represented dynamically as tuples $(cs, is, fun)$ where cs is a
sequence of captured variables (the spore header), is is a sequence of type
class instances, and fun is the spore closure. The only free variables of the
spore closure are the variables in cs.

Creating spores:

\begin{lstlisting}
spore[P] {
  val x_1: T_1 = e_1
  ...
  val x_n: T_n = e_n
  (p: T) => body
}
\end{lstlisting}

A spore has properties $P_1, ..., P_m$ for its captured vars (for simplicity,
above there is only one property P). For now a property can be a type class
like Pickler or CanCapture.

Type classes could be supported as follows. During type checking, implicit
values could be looked up in the standard lexical scope but marked with the
implicit keyword. Type-checking could then rewrite the expression to one which
has the implicit values passed to the spore constructor.

We would rewrite the above expression to the following ($i_1, ..., i_n$ are type
class instances):

\begin{lstlisting}
createSpore[P] {
  val x_1: T_1 = i_1.capture(e_1)
  ...
  val x_n: T_n = i_n.capture(e_n)
  val is = List(i_1, ..., i_n)  // not sure, if we really need those
  (p: T) => body
}
\end{lstlisting}

We might not have to include lookup of implicit values in the formalization.
P[T] could simply return the instance of type class P for type T.

We just need to store those type class instances in the dynamic representation
of spores, so that we have access to them dynamically (since we need to
evaluate the calls to capture).

\subsubsection{Reduction rules}

Reducing spore creation. Spore creation evaluates the spore header which
evaluates the captured expressions $e_1, ..., e_n$, and then passes the results
through the capture methods of the type class instances. The end result is a
sequence of references that we're going to store in the spore tuple (the cs).

Reducing spore application. An application e1 e2 of a spore e1 to an argument
e2 can be reduced by first reducing e1 until it's a reference r1. Then, we
look up the tuple (cs, is, fun) in the heap that reference r1 maps to. Then,
when e2 has also been reduced to a reference r2, we can replace r1 r2 by fun
such that all free vars are replaced by cs and the parameter is replaced by
r2.

\subsubsection{Dynamic representation of spores}

For preservation we then also need to show that $r1$ has the same type as the
\verb|spore| expression. Basically, we need enough dynamic information, so
that we can assign the following type to $r1$:

\begin{lstlisting}
Spore[A, B] {
  type All = P
  type Include = Capture[List[Int]]
  type Exclude = NoCapture[Actor] with NoCapture[Socket]
}
\end{lstlisting}

That means in the opsem, instead of the simple tuple $(cs, is, fun)$ we need to
have a tuple such as
$([{P} ; {List[Int]} ; {Actor, Socket}] A \Rightarrow B, cs, is, fun)$.

Examples for classes that are thread-safe but not shippable: Future/Promise,
Scheduler, ExecutorService, SparkContext (TODO: check). So, we might want to
permit those classes in some spores (those that are not shipped but might be
shared between threads), but not in other spores (those that are shipped).
That's why globally disallowing a type from being captured is not a good idea
and per-spore constraints are better.


\section{Implementation}

We have implemented spores as a macro library for Scala 2.10 and 2.11. Macros are an experimental feature introduced in Scala 2.10 that enable ``macro defs'', methods that take expression trees as arguments and that return an expression tree that is inlined at each invocation site of the macro. Macros are expanded during type checking in a way which enables macros to synthesize their result type during macro expansion, specialized for each site of invocation/expansion.

The implementation for Scala 2.10 requires in addition a compiler plug-in. The plug-in extends type inference for user-defined subclasses of Scala's standard function types which enables infering the types of spore parameters. For Scala 2.11 this plug-in is not required, since this extended type inference is part of Scala's support for Java 8 SAM types (also called ``functional interfaces'').

An expression \verb|spore { val y: S = s; (x: T) => /* body */ }| invokes the \verb|spore| macro which is passed the block \verb|{ val y... }| as an expression tree. A spore without type constraints simply checks that within the body of the spore's closure, only the parameter \verb|x| as well as the variables in the spore header are accessed (see Section~\ref{sec:design}). The expression tree returned by the macro instantiates an anonymous subclass of the abstract \verb|Spore| class that implements its \verb|apply| method (which itself is inherited from the corresponding standard Scala function trait) by applying the spore's closure. The \verb|Spore| class also has abstract type members \verb|Captured| and \verb|Excluded|. The \verb|Captured| type member is defined by the generated anonymous subclass to be a tuple type with the types of all captured variables. If there are no type constraints the \verb|Excluded| type member is defined to be \verb|No[Nothing]|.

Type constraints are implemented as follows. First, invoking the generic \verb|without| macro passing a type argument \verb|T|, say, augments the generated \verb|Spore| subclass by effectivly adding the clause \verb|with No[T]| to the definition of its \verb|Excluded| type member. Second, the existence of additional bounds on the captured types is detected by attempting to infer an implicit value of type \verb|Property[_]|. If such an implicit value can be inferred, a sequence of types specifying type bounds is obtained as follows. Basically, the type of the implicit value is matched against the pattern \verb|Property[t1] with ... with Property[tn]| where the \verb|ti| are type variables. For each type \verb|ti| an implicit value member

\begin{lstlisting}
    implicit val evi: ti[Captured] = implicitly[ti[Captured]]
\end{lstlisting}
\noindent
is generated and added to the anonymous \verb|Spore| subclass.

The spores package also provides an implicit conversion from standard Scala functions to spores. This conversion is implemented as a macro whose expansion fails if the argument function is not a literal, since in this case it's impossible for the macro to check the spore shape/capturing constraints.

Comparison and relationship with Java8 SAM stuff. Cite paper {\em Java SAM Typed Closures: A Sound and Complete Type Inference System for Nominal Types} ~\cite{JavaSAM}


\section{Evaluation}

% program    | LOC | #closures | #converted | LOC changed | #captured vars
%
% funsets    |  99 | 8 + (1)   | 8          | 7           | 9
% forcomp    | 201 | 6 + (2)   | 4          | 4           | 0
% mandelbrot | 325 | 1         | 1          | 9           | 4
% barneshut  | 722 | ??        | ??         | ??          | ??
% pagerank   | ??  | ??        | ??         | ??          | ??


\section{Case Studies}

\subsection{More Robust Distribution of Closures}

Frameworks like MapReduce~\cite{MapReduce} and Apache Spark~\cite{Spark} are designed for processing large datasets in a cluster. The programming models of these frameworks rely on distributing functions used for well-known map/reduce computation patterns.

In Spark, these patterns are directly expressed in Scala using the standard higher-order functions map and reduce applied to an abstraction for distributed collections called ``resilient distributed dataset'' (RDD). However, to avoid unexpected runtime exceptions due to unserializable closures, when passing closures to RDDs programmers must adopt conventions that are subtle and unchecked by the Scala compiler.

The following example shows a typical pattern extracted from a code base used in production:

\begin{lstlisting}
class GenericOp(sc: SparkContext, mapping: Map[String, String]) {
  private var cachedSessions: spark.RDD[Session] = ...

  def doOp(keyList: List[...], ...): Result = {
    val localMapping = mapping

    val mapFun: Session => (List[String], GenericOpAggregator) = { s =>
      (keyList, new GenericOpAggregator(s, localMapping))
    }

    val reduceFun: (GenericOpAggregator, GenericOpAggregator) => GenericOpAggregator =
      { (a, b) => a.merge(b) }

    cachedSessions.map(mapFun).reduceByKey(reduceFun).collectAsMap
  }
}
\end{lstlisting}
\noindent
The \verb|GenericOp| class provides a method \verb|doOp| which performs a compound operation on the RDD \verb|cachedSessions|. \verb|GenericOp| has a parameter of type \verb|SparkContext|, the main entry point for functionality provided by Spark, and a parameter of type \verb|Map[String, String]| used within \verb|doOp|. The main computation of \verb|doOp| is the last expression of its body: a chain of invocations of \verb|map|, \verb|reduceByKey|, and \verb|collectAsMap|. To ensure that the argument closures of \verb|map| and \verb|reduceByKey| are serializable, the code follows two conventions: first, instead of defining \verb|mapFun| and \verb|reduceFun| as methods of class \verb|GenericOp|, they are defined using lambdas stored in local variables. Second, instead of using the mapping parameter directly, it is first copied into a local variable \verb|localMapping|. The reason for the first convention is that in Scala converting a method to a function implicitly captures a reference to the enclosing object. However, \verb|GenericOp| is not serializable, since it refers to a \verb|SparkContext| which is not serializable. The reason for the second convention is that using mapping directly, would result in \verb|mapFun| capturing the \verb|this| reference to be able to access the \verb|mapping| field.

\subsubsection{Applying spores}

Using Spores the above conventions can be enforced by the compiler, avoiding unexpected runtime exceptions. It is sufficient to turn \verb|mapFun| and \verb|reduceFun| into spores:

\begin{lstlisting}
val mapFun: Spore[Session, (List[String], GenericOpAggregator)] = spore {
  val localMapping = mapping
  (s: Session) =>
    (keyList, new GenericOpAggregator(s, localMapping))
}

val reduceFun: Spore[(GenericOpAggregator, GenericOpAggregator), GenericOpAggregator] =
  spore { (a, b) => a.merge(b) }
\end{lstlisting}
\noindent
The spore shape enforces the use of \verb|localMapping| (which is moved from the method body into \verb|mapFun|). Furthermore, there is no possibility of accidentally capturing a reference to the enclosing object in any other way.


\subsection{Safer closures for parallel collections}

Scala's parallel collections provide data-parallel operations for standard collection types like maps and sequences. These data-parallel operations typically take closures as arguments that are applied to all elements of the underlying collection in parallel. Unfortunately, it is easy to create race conditions by capturing mutable objects within such closures, and Scala's type checker does not provide any assistance for avoiding them.

Spores with context bounds allow making the use of parallel collections safer by preventing mutable objects from being captured in closures passed to data-parallel operations. In the following we show how this can be achieved using a custom property for immutable types as well as a wrapper API utilizing spores instead of regular closures.

The first step is to introduce a custom property for immutable types. It consists of a generic trait and an implicit property object:

\begin{verbatim}
object safe {
  trait Immutable[T]
  implicit object immutableProp extends Property[Immutable]
  ...
}
\end{verbatim}
\noindent
The next step is to mark selected types as immutable by defining an implicit object extending the desired list of types, each type wrapped in the \verb|Immutable| type constructor:

\begin{verbatim}
object safe {
  ...
  import scala.collection.immutable.{Map, Set, Seq}
  implicit object collections extends Immutable[Map[_, _]] with
    Immutable[Set[_]] with Immutable[Seq[_]] with ...
}
\end{verbatim}
\noindent
The definitions so far already allow us to create spores that are guaranteed to capture only types \verb|T| for which an implicit of type \verb|Immutable[T]| exists. To ensure that only spores with the appropriate context bound are passed to higher-order methods such as \verb|map| and \verb|flatMap|, we provide a wrapper for \verb|ParIterable[A]|, the supertype of all parallel collections:

\begin{verbatim}
class ImmutWrapper[A](pc: ParIterable[A]) {
  def map[B](s: Spore[A, B])(implicit i: Immutable[s.Captured]) =
    new ImmutWrapper(pc.map(s))
  def flatMap[B](s: Spore[A, Traversable[B]])
                (implicit i: Immutable[s.Captured]) =
    new ImmutWrapper(pc.flatMap(s))
}
\end{verbatim}
\noindent
This wrapper is then used as follows:

\begin{verbatim}
import safe.{immutableProp, collections}

val m: Map[Int, String] = ...
val pcoll = myColl.par
val wrapper = new ImmutWrapper(pcoll)

wrapper.map { elem =>
  if (m.apply(elem)) transform1(elem)  // OK, m is immutable!
  else transform2(elem)
}
\end{verbatim}
\noindent
In the above example, capturing a mutable object (or, in fact, any object of type \verb|T| that does not have an implicit value of type \verb|Immutable[T]|) would lead to a compile-time error, thus preventing a potential data race.

% \section{Use-Cases?}

% We could show in an example-driven (or even paradigm-driven) way how the
% active objects pattern can be implemented on top of spores. And then we can
% show how spores help enforce certain safety properties that are important for
% that pattern. For example, in the active objects pattern it's important to
% either capture only immutable things, or clone things upon capturing.

% Paradigms or patterns built on top of spores:

% \begin{itemize}
% \item Distributed collections like Spark
% \item Active objects
% \item Futures
% \item Hot-swapping actors
% \item Distributed pipelines / distributed streams
% \end{itemize}

\section{Related Work}
\label{sec:related-work}

Non-academic related work: functions in Rust, 3 different types based on
different possibilities for environments \cite{RustFunctions}; functions,
closures, and procedures. Procedures are shippable.

Parallel closures \cite{ParallelClosures}. First known example of closures
with effectively immutable environment.

CloudHaskell \cite{CloudHaskell}. Introduces a type system that rejects anything
that is not static. Too strict.

Haskell distributed parallel Haskell~\cite{HDPH} extends the serializable closures found in CloudHaskell.

C++ 11 comes with a capture syntax. Though, it's not possible to guarantee that construction is correct. If you require certain things from the closure, how is it expressed? If a method declares a parameter type which is a closure, the important thing is that in this parameter type, you can specify the requirements about capturing. So it's ensured that whatever's passed as an argument to the method satisfies these requirements. This isn't possible with C++ 11 closures. \cite{CplusplusLambas}

Active objects also related. \cite{ActiveObjects} (Since no more agents, remove this?)

Clojure comes with the notion of an agent~\cite{Clojure} which is similar to our synchronization mechanism in that in the Clojure model, functions are sent to other agents which manage some sort of mutable, shared state. The spores-agent model focuses on more than managing mutable state. No notion of shippability, fully-focused on multicore single-machine scenario. (Since no more agents, remove this?)

Java RMI~\cite{JavaRMI}

Termite Scheme~\cite{Termite} also has serializable closures.

Distributed Functional Programming in Scheme~\cite{DFPS} also has serializable closures.

Python also can't reliably serialize functions~\cite{PythonPickle}.

Supporting imperative features typically requires complex effect systems~\cite{DPJ}. our paper provides simpler ways to make programming with closures in imperative OO languages safer

Should reference Cloud Types~\cite{CloudTypes}

Lighter-weight alternative approach to some kind of crazy type and effect system.

\section{Conclusion}

\bibliographystyle{abbrv}
\bibliography{bib}

\end{document}