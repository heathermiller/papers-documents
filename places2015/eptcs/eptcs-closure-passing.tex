\documentclass[submission,copyright,creativecommons]{eptcs}
\providecommand{\event}{PLACES 2015} % Name of the event you are submitting to
\usepackage{breakurl}                % Not needed if you use pdflatex only.

\title{Distributed Programming via Safe Closure Passing}
\author{Philipp Haller
\institute{KTH Royal Institute of Technology\\ Stockholm, Sweden}
\email{phaller@kth.se}
\and
Heather Miller
\institute{EPFL\\
Lausanne, Switzerland}
\email{heather.miller@epfl.ch}
}
\def\titlerunning{Safe Closure Passing}
\def\authorrunning{P. Haller \& H. Miller}
\begin{document}
\maketitle

\begin{abstract}
Programming systems incorporating aspects of functional programming, e.g.,
higher-order functions, are becoming increasingly popular for large-scale
distributed programming. New frameworks such as Apache Spark leverage
functional techniques to provide high-level, declarative APIs for in-memory
data analytics, often outperforming traditional ``big data'' frameworks like
Hadoop MapReduce. However, widely-used programming models remain rather
ad-hoc; aspects such as implementation trade-offs, static typing, and semantics
are not yet well-understood. We present a new asynchronous programming model
that has at its core several principles facilitating functional processing of
distributed data. The emphasis of our model is on simplicity, performance,
and expressiveness. The primary means of
communication is by passing functions (closures) to distributed, immutable
data. To ensure safe and efficient distribution of closures, our model
leverages both syntactic and type-based restrictions. We report on a prototype
implementation in Scala. Finally, we present preliminary experimental results
evaluating the performance impact of a static, type-based optimization of serialization.
\end{abstract}

\section{Introduction}

\nocite{*}
\bibliographystyle{eptcs}
\bibliography{bib}
\end{document}
