% This is "sig-alternate.tex" V2.0 May 2012
% This file should be compiled with V2.5 of "sig-alternate.cls" May 2012
%
% This example file demonstrates the use of the 'sig-alternate.cls'
% V2.5 LaTeX2e document class file. It is for those submitting
% articles to ACM Conference Proceedings WHO DO NOT WISH TO
% STRICTLY ADHERE TO THE SIGS (PUBS-BOARD-ENDORSED) STYLE.
% The 'sig-alternate.cls' file will produce a similar-looking,
% albeit, 'tighter' paper resulting in, invariably, fewer pages.
%
% ----------------------------------------------------------------------------------------------------------------
% This .tex file (and associated .cls V2.5) produces:
%       1) The Permission Statement
%       2) The Conference (location) Info information
%       3) The Copyright Line with ACM data
%       4) NO page numbers
%
% as against the acm_proc_article-sp.cls file which
% DOES NOT produce 1) thru' 3) above.
%
% Using 'sig-alternate.cls' you have control, however, from within
% the source .tex file, over both the CopyrightYear
% (defaulted to 200X) and the ACM Copyright Data
% (defaulted to X-XXXXX-XX-X/XX/XX).
% e.g.
% \CopyrightYear{2007} will cause 2007 to appear in the copyright line.
% \crdata{0-12345-67-8/90/12} will cause 0-12345-67-8/90/12 to appear in the copyright line.
%
% ---------------------------------------------------------------------------------------------------------------
% This .tex source is an example which *does* use
% the .bib file (from which the .bbl file % is produced).
% REMEMBER HOWEVER: After having produced the .bbl file,
% and prior to final submission, you *NEED* to 'insert'
% your .bbl file into your source .tex file so as to provide
% ONE 'self-contained' source file.
%
% ================= IF YOU HAVE QUESTIONS =======================
% Questions regarding the SIGS styles, SIGS policies and
% procedures, Conferences etc. should be sent to
% Adrienne Griscti (griscti@acm.org)
%
% Technical questions _only_ to
% Gerald Murray (murray@hq.acm.org)
% ===============================================================
%
% For tracking purposes - this is V2.0 - May 2012

\documentclass{sig-alternate}

\begin{document}
%
% --- Author Metadata here ---
\conferenceinfo{ICSE}{'14 Hyderabad, India}
% \CopyrightYear{2014} % Allows default copyright year (20XX) to be over-ridden - IF NEED BE.
%\crdata{0-12345-67-8/90/01}  % Allows default copyright data (0-89791-88-6/97/05) to be over-ridden - IF NEED BE.
% --- End of Author Metadata ---

\title{Case Study: Teaching Functional Programming en masse (need title)}
%
% You need the command \numberofauthors to handle the 'placement
% and alignment' of the authors beneath the title.
%
% For aesthetic reasons, we recommend 'three authors at a time'
% i.e. three 'name/affiliation blocks' be placed beneath the title.
%
% NOTE: You are NOT restricted in how many 'rows' of
% "name/affiliations" may appear. We just ask that you restrict
% the number of 'columns' to three.
%
% Because of the available 'opening page real-estate'
% we ask you to refrain from putting more than six authors
% (two rows with three columns) beneath the article title.
% More than six makes the first-page appear very cluttered indeed.
%
% Use the \alignauthor commands to handle the names
% and affiliations for an 'aesthetic maximum' of six authors.
% Add names, affiliations, addresses for
% the seventh etc. author(s) as the argument for the
% \additionalauthors command.
% These 'additional authors' will be output/set for you
% without further effort on your part as the last section in
% the body of your article BEFORE References or any Appendices.

\numberofauthors{8} %  in this sample file, there are a *total*
% of EIGHT authors. SIX appear on the 'first-page' (for formatting
% reasons) and the remaining two appear in the \additionalauthors section.
%
\author{
% You can go ahead and credit any number of authors here,
% e.g. one 'row of three' or two rows (consisting of one row of three
% and a second row of one, two or three).
%
% The command \alignauthor (no curly braces needed) should
% precede each author name, affiliation/snail-mail address and
% e-mail address. Additionally, tag each line of
% affiliation/address with \affaddr, and tag the
% e-mail address with \email.
%
% 1st. author
\alignauthor
Heather Miller\\
       \affaddr{EPFL, Switzerland}\\
       \email{heather.miller@epfl.ch}
% 2nd. author
\alignauthor
Lukas Rytz\\
       \affaddr{EPFL, Switzerland}\\
       \email{lukas.rytz@epfl.ch}
% 3rd. author
\alignauthor Martin Odersky\\
       \affaddr{EPFL, Switzerland}\\
       \email{martin.odersky@epfl.ch}
}

\maketitle
\begin{abstract}
Here
\end{abstract}

% A category with the (minimum) three required fields
\category{H.4}{Information Systems Applications}{Miscellaneous}
%A category including the fourth, optional field follows...
\category{D.2.8}{Software Engineering}{Metrics}[complexity measures, performance measures]

\terms{Theory}

\keywords{ACM proceedings, \LaTeX, text tagging}

\section{Introduction}
Intro

Fox and Patterson~\cite{FoxP12}

One interesting aspect of our course is that it did not require a change in
the way typical on-site university courses are organized.

\subsection{Contributions}

\begin{itemize}

\item We present an evaluation of a large-scale survey among participants of a
  MOOC on functional programming. Of the more than 100'000 registered
  participants, we evaluate   survey responses from about 15'000 students on a
  variety of questions such as perceived   difficulty or educational background.

\item A detailed report on the format and organization of a MOOC on functional
  programming with more than 100'000 participants.

\item The design and implementation of an infrastructure used to automatically
  grade programming assignments in the cloud.

\end{itemize}

\section{Course Format}

\section{Tooling and Automated Grading}

\section{Survey Results}

\section{Related Work}

Vihavainen et al. organized a MOOC ("Helsinki MOOC") on introductory computer
science with an emphasis on programming~\cite{VihavainenLK12}. Compared to the
Helsinki MOOC, our course had a number of registered students two orders of
magnitude larger. Moreover, the Helsinki MOOC targets introductory
programming, whereas our course targets advanced programming principles. As a
result, our course was very popular especially among advanced developers who
already have a Bachelor's or Master's degree. The Helsinki MOOC does not treat
university students and MOOC participants equal with respect to the material
used for exercises: the students in their university course are beta testers
of the exercise material; thus, only after this beta test and necessary
adjustments is the material released to non-local MOOC participants. Other
organizational differences exist. For example, they give formal credits for
``apprentices'' who are unpaid ``advisors'' among fellow students with limited
responsibilities. Their Extreme Apprenticeship (XA) learning methodology
required a staff of around 20 persons associated with the course, with
different roles and responsibilities. It is unclear whether the XA methodology
would scale to a course of the scale of progfun (50'000 registered students
versus 500 registered students).


%
% The following two commands are all you need in the
% initial runs of your .tex file to
% produce the bibliography for the citations in your paper.
\bibliographystyle{abbrv}
\bibliography{sigproc}  % sigproc.bib is the name of the Bibliography in this case
% You must have a proper ".bib" file
%  and remember to run:
% latex bibtex latex latex
% to resolve all references
%
% ACM needs 'a single self-contained file'!
%
%APPENDICES are optional
%\balancecolumns
\end{document}
