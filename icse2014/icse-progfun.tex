% This is "sig-alternate.tex" V2.0 May 2012
% This file should be compiled with V2.5 of "sig-alternate.cls" May 2012
%
% This example file demonstrates the use of the 'sig-alternate.cls'
% V2.5 LaTeX2e document class file. It is for those submitting
% articles to ACM Conference Proceedings WHO DO NOT WISH TO
% STRICTLY ADHERE TO THE SIGS (PUBS-BOARD-ENDORSED) STYLE.
% The 'sig-alternate.cls' file will produce a similar-looking,
% albeit, 'tighter' paper resulting in, invariably, fewer pages.
%
% ----------------------------------------------------------------------------------------------------------------
% This .tex file (and associated .cls V2.5) produces:
%       1) The Permission Statement
%       2) The Conference (location) Info information
%       3) The Copyright Line with ACM data
%       4) NO page numbers
%
% as against the acm_proc_article-sp.cls file which
% DOES NOT produce 1) thru' 3) above.
%
% Using 'sig-alternate.cls' you have control, however, from within
% the source .tex file, over both the CopyrightYear
% (defaulted to 200X) and the ACM Copyright Data
% (defaulted to X-XXXXX-XX-X/XX/XX).
% e.g.
% \CopyrightYear{2007} will cause 2007 to appear in the copyright line.
% \crdata{0-12345-67-8/90/12} will cause 0-12345-67-8/90/12 to appear in the copyright line.
%
% ---------------------------------------------------------------------------------------------------------------
% This .tex source is an example which *does* use
% the .bib file (from which the .bbl file % is produced).
% REMEMBER HOWEVER: After having produced the .bbl file,
% and prior to final submission, you *NEED* to 'insert'
% your .bbl file into your source .tex file so as to provide
% ONE 'self-contained' source file.
%
% ================= IF YOU HAVE QUESTIONS =======================
% Questions regarding the SIGS styles, SIGS policies and
% procedures, Conferences etc. should be sent to
% Adrienne Griscti (griscti@acm.org)
%
% Technical questions _only_ to
% Gerald Murray (murray@hq.acm.org)
% ===============================================================
%
% For tracking purposes - this is V2.0 - May 2012

\documentclass{sig-alternate}

\usepackage{listings,xspace}
\usepackage{url}

\lstdefinelanguage{Scala}%
{morekeywords={abstract,case,catch,char,class,%
    def,else,extends,final,%
    if,import,%
    match,module,new,null,object,override,package,private,protected,%
    public,return,super,this,throw,trait,try,type,val,var,with,implicit,%
    macro,sealed,%
  },%
  sensitive,%
  morecomment=[l]//,%
  morecomment=[s]{/*}{*/},%
  morestring=[b]",%
  morestring=[b]',%
  showstringspaces=false%
}[keywords,comments,strings]%

\lstset{language=Scala,%
  mathescape=true,%
  columns=[c]fixed,%
  basewidth={0.5em, 0.40em},%
  basicstyle=\tt,%
  xleftmargin=0.0cm
}

\begin{document}
%
% --- Author Metadata here ---
\conferenceinfo{ICSE}{'14 Hyderabad, India}
% \CopyrightYear{2014} % Allows default copyright year (20XX) to be over-ridden - IF NEED BE.
%\crdata{0-12345-67-8/90/01}  % Allows default copyright data (0-89791-88-6/97/05) to be over-ridden - IF NEED BE.
% --- End of Author Metadata ---

\title{Case Study: Teaching Functional Programming en masse (need title)}
%
% You need the command \numberofauthors to handle the 'placement
% and alignment' of the authors beneath the title.
%
% For aesthetic reasons, we recommend 'three authors at a time'
% i.e. three 'name/affiliation blocks' be placed beneath the title.
%
% NOTE: You are NOT restricted in how many 'rows' of
% "name/affiliations" may appear. We just ask that you restrict
% the number of 'columns' to three.
%
% Because of the available 'opening page real-estate'
% we ask you to refrain from putting more than six authors
% (two rows with three columns) beneath the article title.
% More than six makes the first-page appear very cluttered indeed.
%
% Use the \alignauthor commands to handle the names
% and affiliations for an 'aesthetic maximum' of six authors.
% Add names, affiliations, addresses for
% the seventh etc. author(s) as the argument for the
% \additionalauthors command.
% These 'additional authors' will be output/set for you
% without further effort on your part as the last section in
% the body of your article BEFORE References or any Appendices.

\numberofauthors{8} %  in this sample file, there are a *total*
% of EIGHT authors. SIX appear on the 'first-page' (for formatting
% reasons) and the remaining two appear in the \additionalauthors section.
%
\author{
% You can go ahead and credit any number of authors here,
% e.g. one 'row of three' or two rows (consisting of one row of three
% and a second row of one, two or three).
%
% The command \alignauthor (no curly braces needed) should
% precede each author name, affiliation/snail-mail address and
% e-mail address. Additionally, tag each line of
% affiliation/address with \affaddr, and tag the
% e-mail address with \email.
%
% 1st. author
\alignauthor
Heather Miller\\
       \affaddr{EPFL, Switzerland}\\
       \email{heather.miller@epfl.ch}
% 2nd. author
\alignauthor
Philipp Haller\\
       \affaddr{Typesafe, Switzerland}\\
       \email{philipp.haller@typesafe.com}
% 3rd. author
\alignauthor
Lukas Rytz\\
       \affaddr{EPFL, Switzerland}\\
       \email{lukas.rytz@epfl.ch}
% 4th. author
\alignauthor Martin Odersky\\
       \affaddr{EPFL, Switzerland}\\
       \email{martin.odersky@epfl.ch}
}

\maketitle
\begin{abstract}
Here
\end{abstract}

% A category with the (minimum) three required fields
\category{H.4}{Information Systems Applications}{Miscellaneous}
%A category including the fourth, optional field follows...
\category{D.2.8}{Software Engineering}{Metrics}[complexity measures, performance measures]

\terms{Theory}

\keywords{ACM proceedings, \LaTeX, text tagging}

\section{Introduction}

MOOCs have the potential to revolutionize teaching and learning not only at
universities, but also in industry, in particular, the software industry.
Courses with topics related to software engineering, using relevant
technologies and tools, can be a very attractive vehicle for developing skills
that are invaluable in day-to-day software engineering practice.

This paper describes the experience organizing a MOOC on functional
programming principles with more than 100'000 registered students. What sets
the course apart from other MOOCs related to programming or software
engineering is not only its large number of registered students; perhaps more
importantly, the course has had one of the highest completion rates of any
MOOC worldwide~\cite{Parr13}.\footnote{At the time the first iteration of the
course was completed at the end of November 2012.}

We present results of surveys among more than 12'000 respondents that we
collected for the first two completed iterations of the course. Our results
reveal several surprising trends related to aspects such as the motivation for
participants to take the course, the educational background of participants,
or the perceived difficulty. To the best of our knowledge, results of related
surveys of a comparable size have not been evaluated before.

For example, the well-known MOOC on software engineering organized by Fox and
Patterson~\cite{FoxP12} has had around 50'000 registered students in its first
iteration, and is thus comparable in size to our course; however, no survey of
a comparable size was conducted among participants of the MOOC. Moreover, our
selection of questions provides new insights, related to the interplay of
MOOCs with professional software engineering, in particular.


\subsection{The data}

[Post-course survey results from 6,000-8,000 participants for both iterations]

[Post-course survey results from EPFL students]

[Scores/performance of EPFL students relative to entire course]

% number of respondents:
% Fall 2012:    7,492 (out of ~50,000 registered students)
% Spring 2013:  4,595 (out of ~37,000 registered students)
% Total:       12,087

\paragraph{Contributions} This paper makes the following contributions:

\begin{itemize}

\item A detailed experience report on a large-scale MOOC on functional programming
  principles. At the time of this writing the third iteration of the course is running. So far,
  more than 100'000 participants have registered for the course. Moreover, the course has had
  one of the highest completion rates of any MOOC worldwide.

\item A large data set with all course statistics and survey results evaluated in this paper,
  released as an open-source project~\cite{progfun-stats}. To the best of our knowledge data
  sets on MOOC statistics and survey results of a comparable size have not been published
  before. In addition, the open-source project provides a collection of scripts and tools for
  further analysis and exploration of the collected data through interactive, web-based
  visualizations of the data.

\item An evaluation of course statistics and surveys among more than 12'000 respondents
  collected for the first two completed iterations of the course. Our evaluation uncovers facts
  and correlations which support a new explanation for the course's very high completion rate.
  Moreover, the survey results provide new insights into the role of new technologies and tools
  for effective teaching and learning.

\end{itemize}

\section{Course Format}

The objective of the present MOOC was to introduce functional programming
principles. As such it covered topics such as recursion, persistent data
structures, higher-order functions, and pattern matching. The course evolved
from a regular on-campus course at EPFL for second-year undergraduates based
on material from SICP~\cite{Abelson85}. Throughout the course, the Scala
programming language~\cite{Odersky-Spoon-Venners07} was used, both in the
lectures and in the homework assignments.

The course consisted of both an online part and an offline part, since it was
offered as a regular university course at EPFL for second-year undergraduate
students in Computer Science in addition to the MOOC. The online part comprised all the elements of our MOOC, whereas the offline part comprised
additional elements exclusive to EPFL students.

\subsection{Elements of the MOOC}

\paragraph{Lectures} The course lectures were provided in the form of short
online videos with a length of about 8-12 minutes each. Each week 5-7 videos
were released. In the first iteration of the course all videos came with
transcriptions. Those transcriptions were dropped in the second iteration. The
online video player had controls for speeding up or slowing down playback of
the video. Additionally, each video contained interactive, not-for-credit
quizzes requiring real-time participation of the students.

\paragraph{Assignments}

- weekly or biweekly assignments, all programming exercises.

- unique aspects of the assignment handling: submit assignments via command line, compile remotely, automatic grading: (a) subject submission to a test suite, and (b) analyze with a custom style checker. goal of the style checker: penalize submissions that were obviously not written in the functional style taught in the course.

- The test suite that was used for grading was {\em secret}; only a small scafolding containing a couple of sample tests was provided to the students (correct?). This was an incentive for students to try and create a comprehensive test suite themselves.

- students could submit revisions of their solution as often as they liked
 without penalty until the submission deadline. upon each submission: receive feedback about how their code fared on the secret test suite. This feedback contained hints about failing tests including the hypothetical reduction in points. Additionally, the result of running our style checker was included to give feedback on aspects of style that needed to be improved.

\paragraph{Certificate of completion} Students who obtained more than 60\% of
all possible points received a certificate of completion. In addition, those
who earned more then 80\% of all possible points received a so-called
``certificate of distinction''.

\subsection{Additional elements of the EPFL course}

The EPFL course was a true superset of the MOOC: EPFL students followed the
online lectures and quizzes just like any other participant registered, in our
case, on the Coursera platform. The elements exclusive to EPFL students
consisted of

\begin{itemize}
\item interactive sessions were students could review the course material and
  answer questions; and
\item written exams which accounted for a majority of the final grade on the
  academic record; these exams were essential to satisfy the requirements of
  their degree.
\end{itemize}

One interesting aspect of our course is that it did not require a change in
the way typical on-site university courses are organized.

\subsection{Commercial Offering}

The course format lends itself to additional, commercial offerings. For the
third iteration of our course on functional programming principles, Typesafe
has introduced supervised tutorial sessions accompanying select Coursera
classes. In weekly, one-hour tutorial sessions experts from Typesafe provide
in-depth answers to questions about the course material, and discuss solutions
to homework assignments past the deadline. Tutorial groups are small (10
participants max) in order to meet individual mentoring needs as much as
possible. Tutorial session slots are offered in both European and American
time zones.


\section{Tooling and Automated Grading}

\section{Survey and Data Set}

After the completion of each iteration of our MOOC we sent a survey to all
registered participants. For the Fall 2012 iteration of the course, we
received responses from 7,492 out of about 50,000 registered students. For the
Spring 2013 iteration of the course, we received responses from 4,595 out of
about 37,000 registered students. Thus, we collected results from a total of
12,087 respondents.

The survey consisted of questions such as the following:

\begin{itemize}

\item What's your age group? (Possible choices: 10-17, 18-24, 25-34, 35-44,
  45-54, 55-64, 65+)

\item What country do you live in?

\item What's your highest degree? (Possible choices: No High School (or equivalent), Some High
  School (or equivalent), High School (or equivalent), Some College (or equivalent), Bachelor's
  Degree (or equivalent), Master's Degree (or equivalent), Doctorate Degree (or equivalent),
  Other)

\item Did you finish the course?

\item How difficult did you find the homework assignments? (Possible choices: 1 - Too Easy, 2,
  3, 4, 5 - Too Hard)

\item Where do you plan to apply what you've learned in this course? (Possible choices:
  Personal projects, Individual project at work, Team project at work, University projects, No
  application plans, Attended for general interest)

\item What experience do you have with other programming languages or paradigms? (asked once
  each for Java, C/C++/Objective-C, Python/Ruby/Perl/other scripting language, C\#/.NET,
  JavaScript, \\Haskell/OCaml/ML/F\#, Lisp/Scheme/Clojure, possible choices: No experience /
  not seen it at all, I've seen and understand some code, I have some experience writing code,
  I'm fluent, I'm an expert)

\end{itemize}

% this represents the data that goes into your bar graph
% where the String is the label on the x-axis, and the
% Int is the value for each bar


\begin{figure}[ht!]
  \begin{lstlisting}
object WorthItBarGraph
    extends SimpleBarGraphFactory {
  import CourseraData.worthIt

  val name = "worth-it.html" // file name
  val label = "Percentage"   // label on y axis

  override val width = 250
  override val height = 250
  override val maxy = 70

  def data: List[(String, Int)] = {
    val counts = getFreqs(worthIt)
      .sortBy(_._1)
      .map { case (name, value) =>
        (name.toString, (value.toDouble /
          worthIt.length * 100).round.toInt)
      }

    val correctedLabels: List[(String, Int)] =
      List(("1 Disagree", counts(0)._2)) ++
        counts.drop(1).take(3) ++
        List(("5 Agree",counts(4)._2))

    correctedLabels
  }
}
  \end{lstlisting}
  \caption{Generating a bar graph which
    represents how ``worth it'' the course was for students.}

  \label{fig:bar-chart}
\end{figure}

All survey results have been released as part of the \textsc{progfun-stats} open-source
project~\cite{progfun-stats}. The survey data is available in simple
plain text formats, such  as tab-separated values. Thus, it is easy to process
and analyze the data in any general-purpose programming language. In addition
to the raw data set, \textsc{progfun-stats} provides a library extension for
the Scala programming language~\cite{Odersky-Spoon-Venners07} which makes it
easy to generate interactive, web-based visualizations.

For example, Figure~\ref{fig:bar-chart} shows how to create a simple bar chart
that represents how ``worth it'' the course was for students, on a scale from
1 (disagree) to 5 (agree).


\section{Evaluation}

\subsection{Explaning the high completion rate}

Show how people used the submission system. People were ``hooked'' on submitting solutions.

\begin{itemize}

\item what we can show is how people used the submission system (how often submitted, how
  many points on final submission)

\item we can also show how many people submitted solutions until the end (did the points
  earned by submitting go down or not? If so, by how much?)

\item so, if we can somehow show that people got ``hooked'' on submitting solutions, then
  the setup of allowing multiple (unbounded) submissions with immediate feedback was
  successful: submitting was motivating

\end{itemize}

Examine motivation and background of participants. A lot of the motivation for
people completing the course could come from the fact that they were trying to
learn something that's important for the job, and something that they can't
get anywhere else in a comparable form (indeed, Scala is not yet taught in
many universities). Survey results indicate that about 40\% of respondents
plan to apply the learned knowledge at work. Of those 40\% respondents, we
found that ??\% said that the course was well worth their time (thus, it was
``effective'' for learning). [New correlation] These results also indicate
that MOOCs are an effective training tool for practicing software engineers.

Furthermore, a certificate of completion was issued. Unlike languages
established in industry for many years, there is no standard Scala
certification for developers; the completion certificate could be regarded by
many as the closest substitute.

\subsection{The role of new tools for teaching and learning}

[submission system]

Furthermore, survey results show (see Figure~\ref{fig:ide-usage}) that the use
of the new worksheet technology lead to a doubling in the percentage of use of
the Eclipse IDE compared to a non-course setting. Thus, the worksheet was a
very strong incentive for people to use Eclipse during the course. Again, the
worksheet fundamentally changed the way people interacted with the course
material. The use of Eclipse was voluntary; in fact, many people used other
IDEs such as IntelliJ IDEA. However, even so, the use of Eclipse was so much
higher than usual. Thus, people welcomed the difference in learning that the
worksheet enabled.

\subsection{Educational Background}

[It turns out most participants already have a degree.]

[We also have results for the question: why did you chose to do the course.]

\subsection{Course Completion Rate}

At the time the first iteration of the progfun course was completed, the
course had one of the highest completion rate of any MOOC world-wide [needs
citation: http://www.katyjordan.com/MOOCproject.html].

Could be good to cite, too:~\cite{Parr13}

\section{Related Work}

Vihavainen et al. report on a MOOC ("Helsinki MOOC") on introductory computer
science with an emphasis on programming~\cite{VihavainenLK12}. Compared to the
Helsinki MOOC, our course had a number of registered students two orders of
magnitude larger. Moreover, the Helsinki MOOC targets introductory
programming, whereas our course targets advanced programming principles. As a
result, our course was very popular especially among advanced developers who
already have a Bachelor's or Master's degree. The Helsinki MOOC does not treat
university students and MOOC participants equal with respect to the material
used for exercises: the students in their university course are beta testers
of the exercise material; thus, only after this beta test and necessary
adjustments is the material released to non-local MOOC participants. Other
organizational differences exist. For example, they give formal credits for
``apprentices'' who are unpaid ``advisors'' among fellow students with limited
responsibilities. Their Extreme Apprenticeship (XA) learning methodology
required a staff of around 20 persons associated with the course, with
different roles and responsibilities. It is unclear whether the XA methodology
would scale to a course of the scale of progfun (50'000 registered students
versus 500 registered students).


%
% The following two commands are all you need in the
% initial runs of your .tex file to
% produce the bibliography for the citations in your paper.
\bibliographystyle{abbrv}
\bibliography{sigproc}  % sigproc.bib is the name of the Bibliography in this case
% You must have a proper ".bib" file
%  and remember to run:
% latex bibtex latex latex
% to resolve all references
%
% ACM needs 'a single self-contained file'!
%
%APPENDICES are optional
%\balancecolumns
\end{document}
