We have presented F-P, a new programming model and principled substrate for
building data-centric distributed systems. Built atop a foundation consisting
of performant and type-safe serialization, and safe, serializable closures, we
have shown that it's possible to build elegant fault-tolerant functional
systems. One insight of our model is that lineage-based fault recovery
mechanisms, used in widespread frameworks for distribution, can be modeled
elegantly in a functional way using persistent data structures. Our operational
semantics shows that this approach makes it even amenable to formal treatment.
We have also shown that F-P is able to express patterns of computation richer
than those supported by common ``big data'' frameworks while maintaining
fault-tolerance--such as decentralized peer-to-peer patterns of communication.
Finally, we have implemented our approach in and for Scala, as well as numerous
applications on top, and have discovered new ways to reconcile type-specific
serializers with patterns of static typing common in distributed systems.

A great deal of future work remains. In the short-term, we aim to continue to
build different sorts of distributed frameworks and applications atop F-P in an
effort to work towards a production-ready implementation of our model for
consumption by the Scala community at large.

In the long-term, we plan to better understand concerns of separate compilation
in order to evaluate whether our model could be of help in coordinating between
microservices.\footnote{Microservices are small, independent
  (separately-compiled) services running on different machines which
communicate with each other to together make up a single and complex
application. They are a predominant trend in industry amongst rich and
complicated web-based services.}

% Adapt this model for streaming computation. This would take coming up with a
% solution to null out references in the {\em lineage}. This would make it
% possible to instantiate and populate new silos to handle incoming data.

% The most successful systems for ``big data'' processing have all adopted
% functional APIs. But the innards of these systems are often built atop
% imperative and weakly-typed stacks, which complicates the design and
% implementation of distributed system essentials like fault-tolerance. We
% present a new programming model we call  {\em function passing} designed to
% overcome many of these issues by providing a more principled substrate on
% which to build data-centric distributed systems. A key idea is to pass safe,
% well-typed serializable functions to immutable distributed data.  The F-P
% model itself can be thought of as a distributed persistent functional data
% structure, which stores in its nodes transformations to data rather than the
% distributed data itself.  Thus, the model simplifies failure recovery by
% design--data is recovered by replaying function applications atop immutable
% data loaded from stable storage. Lazy evaluation is also central to our
% model; by carefully incorporating laziness into our design (only at the point
% of initiating network communication), our model remains easy to reason about
% while remaining efficient in time and memory. We formalize our programming
% model in the form of a small-step operational semantics which includes a
% precise specification of the semantics of functional fault recovery. We
% implement our model in and for the Scala programming language, and provide a
% small evaluation of the efficiency of our implementation of the model.

% \section{Future Work}

% Adapt this model for streaming computation. This would take coming up with a
% solution to null out references in the {\em lineage}. This would make it
% possible to instantiate and populate new silos to handle incoming data.
