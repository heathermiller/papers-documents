% THIS IS SIGPROC-SP.TEX - VERSION 3.1
% WORKS WITH V3.2SP OF ACM_PROC_ARTICLE-SP.CLS
% APRIL 2009
%
% It is an example file showing how to use the 'acm_proc_article-sp.cls' V3.2SP
% LaTeX2e document class file for Conference Proceedings submissions.
% ----------------------------------------------------------------------------------------------------------------
% This .tex file (and associated .cls V3.2SP) *DOES NOT* produce:
%       1) The Permission Statement
%       2) The Conference (location) Info information
%       3) The Copyright Line with ACM data
%       4) Page numbering
% ---------------------------------------------------------------------------------------------------------------
% It is an example which *does* use the .bib file (from which the .bbl file
% is produced).
% REMEMBER HOWEVER: After having produced the .bbl file,
% and prior to final submission,
% you need to 'insert'  your .bbl file into your source .tex file so as to provide
% ONE 'self-contained' source file.
%
% Questions regarding SIGS should be sent to
% Adrienne Griscti ---> griscti@acm.org
%
% Questions/suggestions regarding the guidelines, .tex and .cls files, etc. to
% Gerald Murray ---> murray@hq.acm.org
%
% For tracking purposes - this is V3.1SP - APRIL 2009

\documentclass{acm_proc_article-sp}

\usepackage[T1]{fontenc}
\usepackage{amsmath,amssymb,latexsym}
\usepackage{wasysym}
\usepackage{pifont}
\usepackage{graphicx}
\usepackage{fixltx2e}
\usepackage{listings}
\usepackage{mathpartir}
\usepackage[labelfont=bf]{caption}
\usepackage{subcaption}
\captionsetup{compatibility=false}

\newcommand{\bling}{{\small\ding{71}}$^{\tiny\rotatebox[origin=c]{45}{\wasylozenge}}_\text{\ding{75}}$\normalsize bling$_{\tiny\rotatebox[origin=c]{45}{\wasylozenge}}^{~\star}${\small\ding{84}}}


% Member sequences
\newcommand{\seq}[1]{\overline{#1}}

% arrays
\newcommand{\ba}{\begin{array}}
\newcommand{\ea}{\end{array}}
\newcommand{\bda}{\[\ba}
\newcommand{\eda}{\ea\]}
\newcommand{\ei}{\end{array}}
\newcommand{\bcases}{\left\{\begin{array}{ll}}
\newcommand{\ecases}{\end{array}\right.}
\newcommand{\sporeurl}{URL withheld for the sake of anonymity of review.}
% spacing
\newcommand{\gap}{\quad\quad}
\newcommand{\andalso}{\quad\quad}
\newcommand{\biggap}{\quad\quad\quad}
\newcommand{\nextline}{\\ \\}
\newcommand{\htabwidth}{0.5cm}
\newcommand{\tabwidth}{1cm}
\newcommand{\htab}{\hspace{\htabwidth}}
\newcommand{\tab}{\hspace{\tabwidth}}
\newcommand{\linesep}{\ \hrulefill \ \smallskip}

\newcommand{\ie}{{\em i.e.,~}}
\newcommand{\eg}{{\em e.g.,~}}
\newcommand{\etc}{{\em etc}}

\lstdefinelanguage{Scala}%
{morekeywords={abstract,case,catch,char,class,%
    def,else,extends,final,%
    if,import,%
    match,module,new,null,object,override,package,private,protected,%
    public,return,super,this,throw,trait,try,type,val,var,with,implicit,%
    macro,sealed,%
  },%
  sensitive,%
  morecomment=[l]//,%
  morecomment=[s]{/*}{*/},%
  morestring=[b]",%
  morestring=[b]',%
  showstringspaces=false%
}[keywords,comments,strings]%


\begin{document}

\title{Typographic Type Theory and Compiler}

\numberofauthors{2}

\author{
\alignauthor
Heather Miller, PPPoE\\
       \affaddr{Institute for Clarity in Documentation}\\
       \affaddr{Switzerland}
% 2nd. author
\alignauthor
Herr Doktor Klaus Haller, PhD\\
       \affaddr{Die Deutschausgewandertenbundesliga e.V.}\\
       \affaddr{Schweiz}
}


\maketitle
\begin{abstract}

In this work, we provide a hypothetical compiler, which may or may not be
actually implemented, that takes your type judgments and judges them.

We introduce \texttt{typc}, pronounced ``tipsy'', a compiler that's objective
and judgmental. When it doesn't reject your judgments, it suggests types of
its own such as other possible companion types and/or type attributes, {\em
e.g.,} weight, style, condensation, width, slant, italicization, rotation,
waviness, and/or ornamentation. \texttt{typc} may even suggest better colors
for your environment. \end{abstract}

\section{Introduction}

Only designers have design sense. What if compilers could have design sense too?

We introduce \texttt{typc}, a compiler whose taste remains opaque to anyone
outside its self-proclaiming, highly-selective circle. \texttt{typc} seeks to
to diverge from the mainstream and carve a cultural niche all for itself, thus
making it an excellent tool for the practicing computer scientist to use for
design sense.

Give your presentations \bling. Make your thesis look like a.

\section{Preliminaries}

Due to the fact that purveyors of digital computers and compilers frequently
refer to some other harebrained notion of a type, the authors find it
necessary what {\em type} really means.

A {\em type family} is made of {\em types} that share common design features.
Each type has a specific weight, style, condensation, width, slant,
italicization, rotation, waviness, and/or ornamentation.

Programs consist of a list of key-value pairs, $(k, v)$. This list represents
a document template, consisting of type attributes for portions of documents
such as different headings, body text, and bulleted and ordered lists.

Programs that aren't rejected receive suggestions of types to


\section{Foundations}

In order for \texttt{typc} to keep to keep its taste opaque to those outside
of its highly-selective circle, we omit several important judgments on which
\texttt{typc}'s taste is founded. We do, however, provide the {\em ESPN The
Highlights} version of \texttt{typc}'s more notable statics.

\inferrule[]
{ |\Gamma| < 3
}
{ \Gamma \vdash t: T~\textsc{Reject!!}
}


\section{Case Studies}

\section{Conclusion}
Switzerland. Switzerland. Switzerland. Switzerland. Switzerland. Switzerland. Switzerland. Switzerland. Switzerland. Switzerland. Switzerland ist gut.

% \balancecolumns
% That's all folks!
\bibliographystyle{abbrv}
\bibliography{bib}  % sigproc.bib is the name of the Bibliography in this case

\end{document}
