% acmtr.tex
% revised 1/20/97
% updated 06/01/01
% $Header: acmtr.tex,v 1.5 2/14/96 11:07:57 boyland Exp $

\documentclass[acmtocl]{acmtrans2m}
%&t&{\tt #}&
%&v&\verb|#|&

\acmVolume{2}
\acmNumber{3}
\acmYear{01}
\acmMonth{09}

\usepackage[usenames,dvipsnames]{color}
\usepackage{graphicx}
\usepackage{subfig}


\newcommand{\BibTeX}{{\rm B\kern-.05em{\sc i\kern-.025em b}\kern-.08em
    T\kern-.1667em\lower.7ex\hbox{E}\kern-.125emX}}
\newcommand{\smbox}[1]{\mbox{\scriptsize #1}}

%%% Width of all Gnuplot figures
\newlength{\gnuplotWidth}
\setlength{\gnuplotWidth}{.98\columnwidth}

%%% Where figures are located
\graphicspath{{include/}}

\definecolor{HeatherBlue}{rgb}{0,0,0.75}


\markboth{\textcolor{Red}{Miller and Haller}}{\textcolor{Red}{Open Artifacts}}

\title{{\color{Red}Open Artifacts}}
\author{\textcolor{Red}{HEATHER MILLER}\\EPFL \and
\textcolor{Red}{PHILIPP HALLER}\\Typesafe, Switzerland}

% Abstract
% \begin{abstract}
% {\bf Abstract} unnecessary
% \end{abstract}

% \category{}{Research Agenda}{January 2011}%

\begin{document}


%\setcounter{page}{111}
%
%\begin{bottomstuff}
%\end{bottomstuff}
\maketitle


\section{Introduction}

yeah.

\begin{enumerate}

\item \textbf{\textit{An elegant abstraction}} which conceptually facilitates the transition from sequential designs to efficient parallel algorithms, for ML experts.

\item \textbf{\textit{An implementation}} of this framework which handles and hides from the user the ``nitty-gritty" parallelization/distribution details, such as communication protocols, fault-tolerance, partitioning and distribution of data, so that users in various communities with little experience in distributed/parallel programming can fully utilize parallel/distributed architectures.

\end{enumerate}


\bibliographystyle{acmtrans}
\bibliography{bib}
% \begin{received}
% Prepared January 2011
% \end{received}



\end{document}


